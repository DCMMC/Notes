% !TeX encoding = UTF-8
% !TeX program = xelatex
% !TeX spellcheck = <none>

\documentclass[degree=project, degree-type=project]{thuthesis}
\usepackage{mathtools}

% 论文基本配置,加载宏包等全局配置
\thusetup{
    output = electronic,
    title  = {随机过程在机器学习中的应用},
    author  = {肖文韬},
    studentid = {2020214245},
    course = {工程硕士数学},
    include-spine = false,
}

\usepackage{float}
\usepackage[sort]{natbib}
\bibliographystyle{thuthesis-numeric}
\graphicspath{{figures/}}

\begin{document}

% 封面
\maketitle

\frontmatter
\begin{abstract}
    论文的摘要是对论文研究内容和成果的高度概括。摘要应对论文所研究的问题及其研究目
    的进行描述,对研究方法和过程进行简单介绍,对研究成果和所得结论进行概括。摘要应
    具有独立性和自明性,其内容应包含与论文全文同等量的主要信息。使读者即使不阅读全
    文,通过摘要就能了解论文的总体内容和主要成果。
    
    论文摘要的书写应力求精确、简明。切忌写成对论文书写内容进行提要的形式,尤其要避
    免“第 1 章……;第 2 章……;……”这种或类似的陈述方式。
    
    本文介绍清华大学论文模板 \thuthesis{} 的使用方法。本模板符合学校的本科、硕士、
    博士论文格式要求。
    
    本文的创新点主要有:
    \begin{itemize}
        \item 用例子来解释模板的使用方法;
        \item 用废话来填充无关紧要的部分;
        \item 一边学习摸索一边编写新代码。
    \end{itemize}
    
    关键词是为了文献标引工作、用以表示全文主要内容信息的单词或术语。关键词不超过 5
    个,每个关键词中间用分号分隔。(模板作者注:关键词分隔符不用考虑,模板会自动处
    理。英文关键词同理。)
    
    % 关键词用“英文逗号”分隔
    \thusetup{
        keywords = {TeX, LaTeX, CJK, 模板, 论文},
    }
\end{abstract}

% 目录
\tableofcontents

% 插图和附表清单
\listoffigures           % 插图清单

% 正文部分
\mainmatter



% 其他部分
\backmatter

% 参考文献
\bibliography{ref/refs}  % 参考文献使用 BibTeX 编译

% 附录
\appendix

\end{document}