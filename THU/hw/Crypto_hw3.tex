% !TeX encoding = UTF-8
% !TeX program = xelatex
% !TeX spellcheck = en_US

\documentclass[degree=project,degree-type=project,cjk-font=noto]{thuthesis}
\usepackage{mathtools}
\usepackage{tikz}
\usetikzlibrary{shapes,arrows}
\usepackage[autosize]{dot2texi}
% Syntax Highlighting in LaTeX, need pygments
% Must build with xelatex -shell-escape -enable-8bit-chars.
\usepackage{minted}
% https://tex.stackexchange.com/a/112573
\usepackage{tcolorbox}
\usepackage{etoolbox}
\BeforeBeginEnvironment{minted}{\begin{tcolorbox}}%
\AfterEndEnvironment{minted}{\end{tcolorbox}}%
% color for minted
\definecolor{friendlybg}{HTML}{f0f0f0}


% 论文基本配置,加载宏包等全局配置
\thusetup{
    output = electronic,
    title  = {作业三},
    author  = {肖文韬},
    studentid = {2020214245},
    major = {电子信息(计算机技术)},
    email = {xwt20@mails.tsinghua.edu.cn},
    course = {密码学与网络安全},
    include-spine = false,
}


\usepackage{float}
\usepackage[sort]{natbib}
\bibliographystyle{thuthesis-numeric}
\graphicspath{{figures/}}


\setlist[enumerate,1]{label=\arabic*.}
\setlist[enumerate,2]{label=(\alph*)}
\setlist[enumerate,3]{label=\roman*.}
\setlist[enumerate,4]{label=\greek*}


\begin{document}

% 封面
\maketitle

\frontmatter
% \input{data/abstract}

% 目录
% \tableofcontents

% 插图和附表清单
% \listoffiguresandtables
% \listoffigures           % 插图清单

% 正文部分
\mainmatter

\chapter{作业内容}

\begin{enumerate}
  \setlength{\itemsep}{3\parskip}
  \item 在RSA中,如果只知道公钥e和模数N,不知道p、q,求的私钥d的计算复杂度是多少或多大数量集?如果已知p、q呢?请给出证明。
  \newline
  {\heiti 解:}
  

\item 在RSA中:1)给出n=3937和e=17,求私钥d。2)给出n=253和e=3,求私钥d,并对明文m=165进行加密求出密文c。
\newline
{\heiti 解:}
\newline

\end{enumerate}

\end{document}
