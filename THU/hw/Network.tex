% !TeX encoding = UTF-8
% !TeX program = xelatex
% !TeX spellcheck = en_US

\documentclass[degree=project,degree-type=project,cjk-font=noto]{thuthesis}
\usepackage{mathtools}
% Syntax Highlighting in LaTeX, need pygments
% Must build with xelatex -shell-escape -enable-8bit-chars.
\usepackage{minted}
% https://tex.stackexchange.com/a/112573
\usepackage{tcolorbox}
\usepackage{etoolbox}
\BeforeBeginEnvironment{minted}{\begin{tcolorbox}}%
\AfterEndEnvironment{minted}{\end{tcolorbox}}%
% color for minted
\definecolor{friendlybg}{HTML}{f0f0f0}


% 论文基本配置,加载宏包等全局配置
\thusetup{
    output = electronic,
    title  = {光网络前沿研究},
    author  = {肖文韬},
    studentid = {2020214245},
    course = {计算机网络体系结构},
    include-spine = false,
}

\usepackage{float}
\usepackage[sort]{natbib}
\bibliographystyle{thuthesis-numeric}
\graphicspath{{figures/}}

\begin{document}

% 封面
\maketitle

\frontmatter
% \input{data/abstract}

% 目录
\tableofcontents

% 正文部分
\mainmatter

\chapter{引言}

任何技术的发展总是由社会变化的需要和需求来推动的。通信网络从基本的电话网到现在的高速大面积网络的快速发展,是伴随着人们之间交流的社会需求而来,随着用户对新应用需求的不断增加,以及使能技术的进步。现今电信网络的快速变化也是由用户需要随时随地保持联系的需求推动的。新的应用,即多媒体服务、视频会议、互动游戏、互联网服务和万维网,都需要非常大的带宽。除此之外,用户还希望下面的统一网络是可靠的,提供最好的服务,并且性价比高。

因此,我们今天需要的是一个高容量、低成本的通信网络,它要快速、可靠,并能提供从专用服务到最佳服务的各种服务。
现有的传输介质最适合满足这些要求的是 {\heiti 光纤}。
除了拥有太赫兹(约 $10^{12}$ Hz)的巨大带宽,光纤还具有低损耗和低成本的特点。它重量轻、强度高、柔韧性好,而且不受电磁干扰和噪音的影响。 它的安全性和更多的特性使其成为理想的高速传输线路,因此光纤最适合满足当今通信网络的流量要求。 二十世纪末全世界铺设的大量光纤,为今天拥有巨大带宽的光网络信息超级高速公路奠定了基础。

在本文的其余部分,我们将对软件定义光网络进行简要的概述。
此外,我们还将总结机器学习在光网络中的应用。

\chapter{软件定义光网络}
通信网络的增加,传输的数据成千上万,不断的使用网络,除了要避免丢包、延迟、比特率和网络拥堵等各种影响性能的原因外,还需要改善服务和资源,这也是网络的一部分。
曾经只支持电话语音通信的通信网络,现在承载了更多的数据通信,支持高速多媒体服务。
在物理基础设施层面,光网络中现有的光元件现在可以支持多种速度的通信,最高速度可达Tbps,每根光纤在波分复用系统中携带大量的波长。
另外,伴随着光组件基础设施的发展,由于网络采用了智能算法和协议,网络变得更加灵活敏捷,因此现在网络可以轻松应对新的应用和需求。
现在光网络的趋势是向SDON(软件定义光网络)发展,以促进网络操作的可编程性,进一步提高敏捷性,并为用户提供更多的网络功能控制权,从而使新业务和协议的部署更加灵活,网络利用率更高,QoS(服务质量)更好,网络中增加的可读性带来更高的收益,用户可以根据自己的要求管理网络。
现在光网络的模式正在发生转变,SDON是一种快速发展的技术~\cite{SDN,SDON}。

软件定义网络的主要特点是控制平面与数据平面的分离,前者负责管理网络的必要资源与各自的流量,后者由不同的网络设备、传输介质和不同的服务组成。
另一方面,SDN创建的网络是在软件应用的集中控制下工作的,这就不需要使用专门设计的网络设备;这样一来,网络管理员就可以从SDN控制器中塑造流量,而不必接触各个网络设备。
除了来自集中式应用的网络基础设施配置外,关于网络流的传输路径的决策是由SDN做出的,SDN负责有效地响应数据平面,同时考虑到应用平面的状态和请求。这些从控制平面做出的决定被传达给数据平面,并在网络基础设施上进行必要的操作。SDN的每个平面都有其特定的功能和通信协议。

开放网络基金会(ONF)负责制定允许实施SDN的协议或开放标准。在软件定义网络所实现的控制协议中,确定了用于自治系统网络中网关终端之间交换路由信息的BGP协议。每个网络都有一个路由器和各自的网络信息,如路由表,存储了不同连接的路由器的路由,可以到达的ip地址,此外还有一个与路由相关的成本度量,以选择最好的和可用的。另外,还有Netconf协议用于网络的安全管理,选择远程程序调用,用于解决网络设备配置中存在的不同不便。此外,还有MPLS传输配置文件协议(MPLS-TP),它是作为一种网络层技术,作为面向连接的分组交换应用(CO-PS),致力于消除CO-PS应用中不相关的功能,增加关键传输的设备。ONF的目的是开发一个可以从软件控制器进行完全控制并管理网络运行的架构。

OpenFlow协议除了对整个网络进行管理和编程外,还可以协调从一个终端向另一个终端发送数据包的路由,因为在控制平面上,数据平面上数据包迁移的决策是集中的。我们之所以选择这个协议,是因为它的使用范围扩展到了光网络,已经发展到可以从不同的颗粒度上对网络资源进行虚拟分配。在应用平面上可以要求或动态控制网络资源的波长、电路、流量的粒度,根据应用平面的要求无缝地重新配置资源。

\chapter{人工智能在光网络中的应用}

人工智能实体和系统有能力通过模仿生物过程进行类似于学习和决策的操作,特别强调人类的认知过程。诸如虚拟个人助理、智能汽车、购买预测、语音识别或智能家居设备等人工智能应用几乎无处不在,类似的基于人工智能的技术已经在改变我们的日常生活,其方式可以提高人类的生产力、安全或健康,甚至影响到我们娱乐或交流的方式。
举例来说,过去几十年来,通过应用基于人工智能的技术来提高远程通信网络的性能,已经成为一个广泛研究的领域,影响到传输、交换和网络管理等领域。光通信网络和系统并没有置身事外,而是开始采用这一学科走向基于人工智能的光网络,从光子器件到控制和管理。

在本节中,我们将介绍人工智能技术在光网络物理层的应用,即在光传输相关问题上的应用。AI技术可以帮助改善网络设备的配置和操作,光学性能监测,调制格式识别,光纤非线性缓解和自适应控制技术,为锁模光纤激光器提供一个自调谐机制。

\section{掺铒光纤放大器(EDFAs)的操作}

EDFAs是另一个光网络组成部分,人工智能技术已在其上得到广泛的应用。EDFAs是光传输网络的关键要素之一,能够通过在光域中对波分复用通道进行放大来扩展传输的光信号。机器学习技术为EDFA在光纤传输中的操作提供了有效的解决方案,解决了EDFA固有的广泛挑战。

具体来说,Huang等人~\cite{EDFA1} 定义了一个有监督的机器学习(使用径向基函数)的回归问题,从历史数据中学习,对多跨度EDFA网络中功率偏移的信道依赖性进行统计建模。它为系统提供了准确的信道增减策略建议,使信道间的功率差距最小化。随着柔性网格网络的到来,为了提高频谱效率,通常会应用动态碎片化来重新优化有源连接的频谱分配,参考文献~\cite{EDFA2}中对前人的研究进行了扩展,以应对动态变化的频谱配置中的功率偏移问题。岭回归模型用于确定给定子信道的影响大小,并应用逻辑回归来指定贡献是否会导致EDFA后功率间差异的增加或减少。

\section{性能监测}

网络控制和管理的一个挑战是如何适应时变的链路性能参数,如光信噪比(OSNR)、非线性因素、色散(CD)和偏振模色散(PMD)等。本小节分析应用人工智能技术监测上述一些因素的适用性。

对传输光信号的物理参数进行估计和采集,可以进行网络诊断,以便对故障采取相应的措施(修复损坏、驱动补偿器/均衡器或在非最佳链路周围重新分配流量)。例如,Wu等人~\cite{Wu}提出了人工神经网络在光学性能监测(OPM)中应用的广泛研究,其中包括从眼图和眼直方图参数中同时识别累积的非线性、OSNR、CD和PMD。该研究采用基于主成分分析的模式识别对异步延迟图进行研究,并在同时监测线性损伤方面产生了准确的结果。

另一项最近的工作面对前面提到的研究的有限的可扩展性,其中提出了一种深度神经网络(DNN),用相干接收器异步采样的原始数据进行训练,用于OSNR监测。结果表明,OSNR的估计是准确的。然而,这种DNN需要配置至少5层,并且需要用40万个样本进行训练才能获得准确的结果,需要较长的训练时间。另外,Thrane等人~\cite{osnr}提出了一个OSNR估计器和一个调制格式分类器,用于采用高级调制格式(最高64QAM)和直接检测的系统。OSNR估计器采用神经网络,而调制格式分类器则采用支持向量机(SVM),两者都是为了学习一个连续的映射函数between输入特征,分别从光电探测器后的功率眼图中提取,以及参考OSNR和调制格式。虽然在OSNR估计和调制格式分类方面得到了准确的结果,但该研究只考虑了白色高斯噪声,而暂时忽略了线性和非线性光纤损伤。

% 其他部分
\backmatter

% 参考文献
\bibliography{ref/refs}  % 参考文献使用 BibTeX 编译

% 附录
\appendix
\end{document}
