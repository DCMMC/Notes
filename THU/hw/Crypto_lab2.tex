% !TeX encoding = UTF-8
% !TeX program = xelatex
% !TeX spellcheck = en_US

\documentclass[degree=project,degree-type=project,cjk-font=noto]{thuthesis}
\usepackage{mathtools}
\usepackage{tikz}
\usetikzlibrary{shapes,arrows}
\usepackage[autosize]{dot2texi}
% Syntax Highlighting in LaTeX, need pygments
% Must build with xelatex -shell-escape -enable-8bit-chars.
\usepackage{minted}
% https://tex.stackexchange.com/a/112573
\usepackage{tcolorbox}
\usepackage{etoolbox}
\BeforeBeginEnvironment{minted}{\begin{tcolorbox}}%
\AfterEndEnvironment{minted}{\end{tcolorbox}}%
% color for minted
\definecolor{friendlybg}{HTML}{f0f0f0}


% 论文基本配置,加载宏包等全局配置
\thusetup{
    output = electronic,
    title  = {实验二},
    author  = {肖文韬},
    studentid = {2020214245},
    major = {电子信息(计算机技术)},
    email = {xwt20@mails.tsinghua.edu.cn},
    course = {密码学与网络安全},
    include-spine = false,
}


\usepackage{float}
\usepackage[sort]{natbib}
\bibliographystyle{thuthesis-numeric}
\graphicspath{{figures/}}


\setlist[enumerate,1]{label=\arabic*.}
\setlist[enumerate,2]{label=(\alph*)}
\setlist[enumerate,3]{label=\roman*.}
\setlist[enumerate,4]{label=\greek*}

\newcommand*\justify{%
  \fontdimen2\font=0.4em% interword space
  \fontdimen3\font=0.2em% interword stretch
  \fontdimen4\font=0.1em% interword shrink
  \fontdimen7\font=0.1em% extra space
  \hyphenchar\font=`\-% allowing hyphenation
}

\renewcommand{\texttt}[1]{%
  \begingroup
  \ttfamily
  \begingroup\lccode`~=`/\lowercase{\endgroup\def~}{/\discretionary{}{}{}}%
  \begingroup\lccode`~=`[\lowercase{\endgroup\def~}{[\discretionary{}{}{}}%
  \begingroup\lccode`~=`.\lowercase{\endgroup\def~}{.\discretionary{}{}{}}%
  \catcode`/=\active\catcode`[=\active\catcode`.=\active
  \justify\scantokens{#1\noexpand}%
  \endgroup
}


\begin{document}

% 封面
\maketitle

\frontmatter
% \input{data/abstract}

% 目录
% \tableofcontents

% 插图和附表清单
% \listoffiguresandtables
% \listoffigures           % 插图清单

% 正文部分
\mainmatter

\chapter{实验介绍}
RSA加密算法是应用最广泛的公钥加密算法,本次实验实现基于RSA算法的加解密以及数字签名功能,包含以下4种操作:生成密钥对、公钥加密、私钥解密和数字签名。
本次实验中密钥长度为2048比特。
实验使用 Rust 语言实现 RSA-2048 的密钥生成,加密解密,以及数字签名操作。

实验目的:

\begin{enumerate}
    \item 熟悉 RSA 公钥加密算法的思路
    \item 学习 RSA 实现上的技巧
    \item 学习 rust 语言的基本用法
\end{enumerate}

实验平台:

\begin{enumerate}
    \item rust 语言
    \item Arch Linux (不依赖具体操作系统,rust 亦可在 windows/macOS 上使用)
\end{enumerate}

\chapter{实验内容}

\section{生成密钥对}

    密钥对的生成过程包括选取随机数,对随机数进行素性测试,根据素数p和q计算n,随机选择和n的欧拉函数互质的e,计算e的逆元d。
选取的大素数p和q应当满足现有的安全性要求,且至少使用两种不同的算法进行素性测试,请在实验报告中说明你选择参数和算法的安全性以及效率。
选取的e同样应当满足安全性要求,至少使用两种不同的算法进行计算逆元d。

\section{素性测试}

本实现采用了两种最主流的概率素性测试算法:

\begin{enumerate}
  \item \textbf{Miller-Rabin 测试},作为费马定理的扩展,每一轮 MR 测试的伪素数的可能性为 $\frac{1}{4}$,所以 $k$ 轮通过仍然是伪素数的可能性为 $4^{-k}$。复杂度 $O(k \log^2 n)$ ($k$ 为测试轮数)。
  \item \textbf{Baillie–PSW 测试}, 结合了 Miller-Rabin 测试和强 Lucas 概率测试。复杂度为 $O(log^2 n)$,低于 MR 测试。
\end{enumerate}

\subsection{模逆运算}

本实现中素数 p 和 q 均为 2048 位,是目前主流的 RSA-2048 实现,密钥长度符合安全要求。

对于 e 的选择,过小的 e(例如 3)会存在安全问题,同时短比特长度和小的 Hamming 权重能够使得加密的效率更加高,目前 OpenSSL 以及其他实现广泛采用的是 65537 (0x10001).
本实现参考主流实现,e 的选择也是 65537.

d 作为 e 的模 $\phi(n)$ 逆元,因为 n 为 4096 位,所以 d 的强度也能够得到保证。

本实现共有两种模逆的算法实现:

\begin{enumerate}
  \item \textbf{扩展欧几里得算法},算法效率 $O(2 \log_{10}(\phi(n)))$(除法运算)。
  \item \textbf{平方幂算法(binary exponentiation)},算法效率 $O(\log(\phi(n)))$。但是该算法只适用于 $\phi(n)$ 为素数的情况。
\end{enumerate}

代码:

  \begin{minted}[texcomments,tabsize=2,fontsize=\normalsize,style=friendly,bgcolor=friendlybg]{rust}
fn modular_inverse(e: &Integer, phi_n: &Integer, method: &str) {
    if method == "extend_gcd" {
        let mut t = Integer::from(0);
        let mut newt = Integer::from(1);
        let mut r = Integer::from(phi_n);
        let mut newr = Integer::from(e);
        let mut quotient = Integer::new();
        let mut tmp = Integer::new();
        while newr.significant_bits() != 0 {
            quotient.assign(&r / &newr);
            tmp.assign(&quotient * &newt);
            tmp *= -1; tmp += &t; t.assign(&newt);
            newt.assign(&tmp);
            tmp.assign(&quotient * &newr);
            tmp *= -1; tmp += &r; r.assign(&newr);
            newr.assign(&tmp);
        }
        if r > 1 { panic!("e is not invertible!"); }
        if t < 0 { t += phi_n; }
        return t;
    } else if method == "binary_exp" {
        if baillie_psw(phi_n, true) {
            println!("phi_n is prime, use binary_exp");
            return Integer::from((&e).pow_mod_ref(
                    &Integer::from(phi_n - 2), &phi_n).unwrap())
        } else {
            return Integer::from((&e).invert_ref(&phi_n).unwrap());
        }
    }
}
  \end{minted}


\begin{figure}[h]
\centering%
\includegraphics[width=\linewidth]{aes_t4.png}
  \caption{加密机和解密机运行结果}
  \label{fig:t4}
\end{figure}

% 其他部分
\backmatter

% 参考文献
\bibliography{ref/refs}  % 参考文献使用 BibTeX 编译

\end{document}
