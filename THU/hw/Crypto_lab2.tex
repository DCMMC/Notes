% !TeX encoding = UTF-8
% !TeX program = xelatex
% !TeX spellcheck = en_US

\documentclass[degree=project,degree-type=project,cjk-font=noto]{thuthesis}
\usepackage{mathtools}
\usepackage{tikz}
\usetikzlibrary{shapes,arrows}
\usepackage[autosize]{dot2texi}
% Syntax Highlighting in LaTeX, need pygments
% Must build with xelatex -shell-escape -enable-8bit-chars.
\usepackage{minted}
% https://tex.stackexchange.com/a/112573
\usepackage{tcolorbox}
\usepackage{etoolbox}
\BeforeBeginEnvironment{minted}{\begin{tcolorbox}}%
\AfterEndEnvironment{minted}{\end{tcolorbox}}%
% color for minted
\definecolor{friendlybg}{HTML}{f0f0f0}


% 论文基本配置,加载宏包等全局配置
\thusetup{
    output = electronic,
    title  = {实验二},
    author  = {肖文韬},
    studentid = {2020214245},
    major = {电子信息(计算机技术)},
    email = {xwt20@mails.tsinghua.edu.cn},
    course = {密码学与网络安全},
    include-spine = false,
}


\usepackage{float}
\usepackage[sort]{natbib}
\bibliographystyle{thuthesis-numeric}
\graphicspath{{figures/}}


\setlist[enumerate,1]{label=\arabic*.}
\setlist[enumerate,2]{label=(\alph*)}
\setlist[enumerate,3]{label=\roman*.}
\setlist[enumerate,4]{label=\greek*}

\newcommand*\justify{%
  \fontdimen2\font=0.4em% interword space
  \fontdimen3\font=0.2em% interword stretch
  \fontdimen4\font=0.1em% interword shrink
  \fontdimen7\font=0.1em% extra space
  \hyphenchar\font=`\-% allowing hyphenation
}

\renewcommand{\texttt}[1]{%
  \begingroup
  \ttfamily
  \begingroup\lccode`~=`/\lowercase{\endgroup\def~}{/\discretionary{}{}{}}%
  \begingroup\lccode`~=`[\lowercase{\endgroup\def~}{[\discretionary{}{}{}}%
  \begingroup\lccode`~=`.\lowercase{\endgroup\def~}{.\discretionary{}{}{}}%
  \catcode`/=\active\catcode`[=\active\catcode`.=\active
  \justify\scantokens{#1\noexpand}%
  \endgroup
}


\begin{document}

% 封面
\maketitle

\frontmatter
% \input{data/abstract}

% 目录
% \tableofcontents

% 插图和附表清单
% \listoffiguresandtables
% \listoffigures           % 插图清单

% 正文部分
\mainmatter

\chapter{实验介绍}
RSA加密算法是应用最广泛的公钥加密算法,本次实验实现基于RSA算法的加解密以及数字签名功能,包含以下4种操作:生成密钥对、公钥加密、私钥解密和数字签名。
本次实验中密钥长度为2048比特。
实验使用 Rust 语言实现 RSA-2048 的密钥生成,加密解密,以及数字签名操作。

实验目的:

\begin{enumerate}
    \item 熟悉 RSA 公钥加密算法的思路
    \item 学习 RSA 实现上的技巧
    \item 学习 rust 语言的基本用法
\end{enumerate}

实验平台:

\begin{enumerate}
    \item rust 语言
\end{enumerate}

\chapter{实验内容}

\section{生成密钥对}

密钥对的生成过程包括选取随机数,对随机数进行素性测试,根据素数p和q计算n,随机选择和n的欧拉函数互质的e,计算e的逆元d。
选取的大素数p和q应当满足现有的安全性要求,且至少使用两种不同的算法进行素性测试,请在实验报告中说明你选择参数和算法的安全性以及效率。
选取的e同样应当满足安全性要求,至少使用两种不同的算法进行计算逆元d。

\subsection{素性测试}

本实现采用了两种最主流的概率素性测试算法:

\begin{enumerate}
  \item \textbf{Miller-Rabin 测试},作为费马定理的扩展,每一轮 MR 测试的伪素数的可能性为 $\frac{1}{4}$,所以 $k$ 轮通过仍然是伪素数的可能性为 $4^{-k}$。复杂度 $O(k \log^2 n)$ ($k$ 为测试轮数)。
  \item \textbf{Baillie–PSW 测试}, 结合了 Miller-Rabin 测试和强 Lucas 概率测试。复杂度为 $O(log^2 n)$,低于 MR 测试。
\end{enumerate}

值得一提的是,素性测试的一些优化技巧:

\begin{enumerate}
  \item 首先测试是否是常用的小素数的倍数,具体来说,RSA-2048 会首先判断随机数是否是自然数中前 384 个素数的倍数。该技巧和 384 的来源参考的是 OpenSSL 最新代码的 \texttt{bn\_prime.c\#L74:12}。
  \item 如果当然随机数不是素数,会对随机数 +2 再判断是否是素数,因为按照素数定理,对于 RSA-2048,连续的数中出现素数的概率为 $\frac{1}{\log(2^{2048})} \approx \frac{1}{1418}$。最坏情况下尝试 700 多次就一定会遇到素数。
\end{enumerate}

miller rabin 测试的代码:

\begin{minted}[texcomments,tabsize=2,fontsize=\normalsize,style=friendly,bgcolor=friendlybg]{rust}
fn miller_rabin_test(rnd: &Integer, iteration: u8, base_2: bool) {
    let n = Integer::from(rnd);
    let mut d: Integer = Integer::from(&n - 1);
    let mut r = 0;
    while !(&d.get_bit(0)) { d >>= 1; r += 1; }
    let mut rng = RandState::new();
    'witness_loop: for _ in 0..iteration {
        let mut n_sub = Integer::from(&n - 2u8);
        let a = if base_2 {
            Integer::from(2)
        } else {
            Integer::from(n_sub.random_below_ref(&mut rng)) + 2u8
        };
        let mut x = a.pow_mod(&d, &n).unwrap();
        n_sub += 1u8;
        if &x == &1 || &x == &n_sub {
            continue 'witness_loop;
        }
        for _ in 1..r {
            x = x.pow_mod(&Integer::from(2), &n).unwrap();
            if &x == &n_sub {
                continue 'witness_loop;
            }
        }
        return false;
    }
    true
}
\end{minted}

因为囿于篇幅,具体的代码注释参见源码。

\subsection{模逆运算}

本实现中素数 p 和 q 均为 2048 位,是目前主流的 RSA-2048 实现,密钥长度符合安全要求。

对于 e 的选择,过小的 e(例如 3)会存在安全问题,同时短比特长度和小的 Hamming 权重能够使得加密的效率更加高,目前 OpenSSL 以及其他实现广泛采用的是 65537 (0x10001).
本实现参考主流实现,e 的选择也是 65537.

d 作为 e 的模 $\phi(n)$ 逆元,因为 n 为 4096 位,所以 d 的强度也能够得到保证。

本实现共有两种模逆的算法实现:

\begin{enumerate}
  \item \textbf{扩展欧几里得算法},算法效率 $O(2 \log_{10}(\phi(n)))$(除法运算)。gcd 的计算复杂度推导很复杂,具体可以参考 TAOCP。
  \item \textbf{平方幂算法(binary exponentiation)},算法效率 $O(\log(\phi(n)))$。但是该算法只适用于 $\phi(n)$ 为素数的情况。
\end{enumerate}

两种模逆的算法实现代码:

  \begin{minted}[texcomments,tabsize=2,fontsize=\normalsize,style=friendly,bgcolor=friendlybg]{rust}
fn modular_inverse(e: &Integer, phi_n: &Integer, method: &str) {
    if method == "extend_gcd" {
        let mut t = Integer::from(0);
        let mut newt = Integer::from(1);
        let mut r = Integer::from(phi_n);
        let mut newr = Integer::from(e);
        let mut quotient = Integer::new();
        let mut tmp = Integer::new();
        while newr.significant_bits() != 0 {
            quotient.assign(&r / &newr);
            tmp.assign(&quotient * &newt);
            tmp *= -1; tmp += &t; t.assign(&newt);
            newt.assign(&tmp);
            tmp.assign(&quotient * &newr);
            tmp *= -1; tmp += &r; r.assign(&newr);
            newr.assign(&tmp);
        }
        if r > 1 { panic!("e is not invertible!"); }
        if t < 0 { t += phi_n; }
        return t;
    } else if method == "binary_exp" {
        if baillie_psw(phi_n, true) {
            println!("phi_n is prime, use binary_exp");
            return Integer::from((&e).pow_mod_ref(
                    &Integer::from(phi_n - 2), &phi_n).unwrap())
        } else {
            return Integer::from((&e).invert_ref(&phi_n).unwrap());
        }
    }
}
  \end{minted}

\subsection{密钥生成}

密钥生成就是通过两个大随机素数 $p,q$ 相乘计算出 $n$,然后再计算 $\phi(n) = (p-1)(q-1)$,然后通过 $e = 65537$ 计算它的模逆 $de = 1 (\text{mod } \phi(n))$。
代码如下:

  \begin{minted}[texcomments,tabsize=2,fontsize=\normalsize,style=friendly,bgcolor=friendlybg]{rust}
fn rsa_key_phase1() -> (Integer, Integer, Integer) {
    let p = generate_prime("miller_rabin");
    let q = generate_prime("miller_rabin");
    let n = Integer::from(&p * &q);
    let phi_n = (p - 1) * (q - 1);
    let e = Integer::from(65537);
    (n, phi_n, e)
}

fn rsa_key_pair(method: &str) -> (Integer, Integer, Integer) {
    let (n, phi_n, e) = rsa_key_phase1();
    let d = modular_inverse(&e, &phi_n, method);
    (n, e, d)
}
\end{minted}

\subsection{运行结果}

\begin{figure}[h]
\centering%
\includegraphics[width=\linewidth]{rsa_t1.png}
  \caption{密钥生成运行结果}
  \label{fig:t1}
\end{figure}

RSA 的实现跟之前 AES 的实现都使用 cargo 组织项目代码。
编译运行的代码:

  \begin{minted}[texcomments,tabsize=2,fontsize=\normalsize,style=friendly,bgcolor=friendlybg]{bash}
  cd target/release && cargo build --release && ./crypto
  \end{minted}

密钥生成的 $n, e, d$ 如图~\ref{fig:t1} 所示,其中 $n$ 是两个 2048 位随机大素数 $p, q$ 的乘积, $e$ 则是固定的。

\section{公钥加密}

本题使用公钥对明文进行加密,明文为 'Cryptography and Network Security;学号;姓名拼音' ,使用自己的学号和姓名拼音替代对应位置字符串。加密过程包括分块,字符串转数字,公钥加密。分块将待加密的字符串分解成多个块,对于每个块分别加密。每个块的长度取决于选择的大素数p和q的大小,具体来说,比特串的长度应当小于log2n。块长度不足的部分需要使用填充字符补齐。字符串转数字可以先转化成比特串,再转化为数字,或者采用自行规定的转化规则。公钥加密部分请使用快速幂算法计算密文。输出每一步的结果。

\subsection{快速幂算法}
快速幂算法就是对计算 $m^e$ 的优化,就是把 $e$ 以二进制表示,复杂度 $O(2\log(n))$.

\subsection{加密填充字符}

首先计算 $\log_2 n$ 作为加密块的大小,因为 $n$ 为 4096 位,也就是每一块为 512 字节。
对于明文的最后一块不足 512 字节的话,在最后面填充 0.
对于每一块的加密,就是把这一块当做一个大整数 $m$, 然后计算 $m^e \text{ mod } n$.
其中 $e, n$ 为公钥对。
代码如下:

  \begin{minted}[texcomments,tabsize=2,fontsize=\normalsize,style=friendly,bgcolor=friendlybg]{rust}
fn rsa_encrypt(key: (&Integer, &Integer), plaintext: &str) -> Vec<u8> {
    let (n, e)  = key;
    let block_size = (n.significant_bits() as f64 / 8.0).ceil() as usize;
    let mut padded_bytes: Vec<u8> = plaintext.to_string().into_bytes();
    let mut pad_size = padded_bytes.len() % block_size;
    if pad_size != 0 {
        pad_size = block_size - pad_size;
    }
    println!("padding_size={}", pad_size);
    padded_bytes.extend(vec![PAD_BYTE; pad_size]);
    for block in padded_bytes.chunks_mut(block_size) {
        let mut num_str = block.iter()
            .map(|x| format!("{:02X}", x))
            .fold(String::from(""),
                  |res: String, curr: String| res + &curr
            );
        let mut m = Integer::from(Integer::parse_radix(
            num_str, 16).unwrap());
        m = quick_pow_mod(m, e, n);
        let mut digits = m.to_string_radix(16);
        if digits.len() % 2 == 1 { digits.insert(0, '0'); }
        for idx in 0..block.len() {
            block[idx] = u8::from_str_radix(&digits[(idx*2)..(idx*2+2)],
              16).unwrap();
        }
    }
    padded_bytes
}
\end{minted}

\subsection{运行结果}

\begin{figure}[h]
\centering%
\includegraphics[width=\linewidth]{rsa_t1.png}
  \caption{公钥加密运行结果}
  \label{fig:t2}
\end{figure}

\section{私钥解密}

私钥解密是公钥加密操作的逆操作,首先需要将比特串转化为数字,然后对数字使用私钥d进行解密,同样建议使用快速幂等算法计算,然后将数字还原成字符串,输出每一步的结果。

\subsection{代码说明}

因为 RSA 加密解密差不多一模一样,而且都能简单。
解密过程就是就是计算 $c^d \text{ mod } n$,其中 $c, d, n$ 分别为密文(块),私钥对。
去除填充字符的过程也比较简单,因为填充的字符就是 0,所以只需要把最后一块最后面连续的 0 去掉就好啦。


% 其他部分
\backmatter

% 参考文献
\bibliography{ref/refs}  % 参考文献使用 BibTeX 编译

\end{document}
