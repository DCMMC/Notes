% !TeX encoding = UTF-8
% !TeX program = xelatex
% !TeX spellcheck = en_US

\documentclass[degree=project,degree-type=project,cjk-font=noto]{thuthesis}
\usepackage{mathtools}
\usepackage{tikz}
\usetikzlibrary{shapes,arrows}
\usepackage[autosize]{dot2texi}
% Syntax Highlighting in LaTeX, need pygments
% Must build with xelatex -shell-escape -enable-8bit-chars.
\usepackage{minted}
% https://tex.stackexchange.com/a/112573
\usepackage{tcolorbox}
\usepackage{etoolbox}
\BeforeBeginEnvironment{minted}{\begin{tcolorbox}}%
\AfterEndEnvironment{minted}{\end{tcolorbox}}%
% color for minted
\definecolor{friendlybg}{HTML}{f0f0f0}


% 论文基本配置,加载宏包等全局配置
\thusetup{
    output = electronic,
    title  = {实验三:基于 PPG 的语音转换系统},
    author  = {肖文韬},
    studentid = {2020214245},
    course = {语音信号数字处理},
    include-spine = false,
}

\usepackage{float}
\usepackage[sort]{natbib}
\bibliographystyle{thuthesis-numeric}
\graphicspath{{figures/}}


\begin{document}

% 封面
\maketitle

\frontmatter
% \input{data/abstract}

% 目录
\tableofcontents

% 插图和附表清单
\listoffigures           % 插图清单

% 正文部分
\mainmatter

\chapter{任务一: 提取 PPG 与声学参数 ($15"$)}

\section{任务介绍}

为了进行语音转换,我们首先需要使用 ASR 系统将源音频转换为一种中间特征(在本实验中就是音素序列 PPG~\cite{PPG}),对每一帧的 MFCC 特征 $X_t$ 我们可以得到所有音素(音素集 $\mathcal{S}$)的后验概率 $\{p(s | X_t) | s \in \mathcal{S}\}$。
同时,我们还可以将原始波形序列加窗得到语音帧,对语音帧进行离散傅里叶变换后,计算各频率分量的能量后可以得到语谱图(线性谱)。
而我们知道人类对低频成分更加敏感,而对高频不敏感,所以我们取对数后可以得到对应的 Mel 谱。
本任务就是使用预训练模型得到音频的 PPG,同时还需要计算得到基频 $F_0$, 线性谱,Mel 谱等声学参数。

接下来的小节就是回答问题啦。

\section{提取音素后验概率 PPG ($4"$)}

\textbf{(1) 简要说明 PPG 提取器(ppg\_extractor)的网络结构,给出网络的基本结构图。}

\begin{figure}[!htp]
\centering%
\includegraphics[width=.8\linewidth]{PPG.png}
  \caption{PPG 提取流程图}
  \label{fig:ppg}
\end{figure}


\textbf{答}: PPG 提取器网络由卷积层、LSTM 和线性层组成,具体组成如图~\ref{fig:ppg}所示。

\chapter{任务二: 训练并测试特定目标说话人的语音转换模型($40"$)}

\chapter{任务三: 探究残差网络对转换性能的影响($15"$)}

\chapter{任务四: 增加说话人嵌入网络,实现多目标说话人的语音转换($20"$)}

% 其他部分
\backmatter

% 参考文献
\bibliography{ref/refs}  % 参考文献使用 BibTeX 编译

% 附录
\appendix
\end{document}
