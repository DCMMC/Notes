% !TeX encoding = UTF-8
% !TeX program = xelatex
% !TeX spellcheck = en_US

\documentclass[degree=project,degree-type=project,cjk-font=noto]{thuthesis}
\usepackage{mathtools}
\usepackage{tikz}
\usetikzlibrary{shapes,arrows}
\usepackage[autosize]{dot2texi}
% Syntax Highlighting in LaTeX, need pygments
% Must build with xelatex -shell-escape -enable-8bit-chars.
\usepackage{minted}
% https://tex.stackexchange.com/a/112573
\usepackage{tcolorbox}
\usepackage{etoolbox}
\BeforeBeginEnvironment{minted}{\begin{tcolorbox}}%
\AfterEndEnvironment{minted}{\end{tcolorbox}}%
% color for minted
\definecolor{friendlybg}{HTML}{f0f0f0}


% 论文基本配置,加载宏包等全局配置
\thusetup{
    output = electronic,
    title  = {实验一},
    author  = {肖文韬},
    studentid = {2020214245},
    major = {电子信息(计算机技术)},
    email = {xwt20@mails.tsinghua.edu.cn},
    course = {密码学与网络安全},
    include-spine = false,
}


\usepackage{float}
\usepackage[sort]{natbib}
\bibliographystyle{thuthesis-numeric}
\graphicspath{{figures/}}


\setlist[enumerate,1]{label=\arabic*.}
\setlist[enumerate,2]{label=(\alph*)}
\setlist[enumerate,3]{label=\roman*.}
\setlist[enumerate,4]{label=\greek*}

\newcommand*\justify{%
  \fontdimen2\font=0.4em% interword space
  \fontdimen3\font=0.2em% interword stretch
  \fontdimen4\font=0.1em% interword shrink
  \fontdimen7\font=0.1em% extra space
  \hyphenchar\font=`\-% allowing hyphenation
}

\renewcommand{\texttt}[1]{%
  \begingroup
  \ttfamily
  \begingroup\lccode`~=`/\lowercase{\endgroup\def~}{/\discretionary{}{}{}}%
  \begingroup\lccode`~=`[\lowercase{\endgroup\def~}{[\discretionary{}{}{}}%
  \begingroup\lccode`~=`.\lowercase{\endgroup\def~}{.\discretionary{}{}{}}%
  \catcode`/=\active\catcode`[=\active\catcode`.=\active
  \justify\scantokens{#1\noexpand}%
  \endgroup
}


\begin{document}

% 封面
\maketitle

\frontmatter
% \input{data/abstract}

% 目录
% \tableofcontents

% 插图和附表清单
% \listoffiguresandtables
% \listoffigures           % 插图清单

% 正文部分
\mainmatter

\chapter{实验介绍}
AES加密算法是迭代型分组密码算法,涉及4种操作:字节替代、行移位、列混合和轮密钥加。
本次实验中密钥长度和分组长度都为128比特,加密轮数为10轮。
实验使用 Rust 语言实现 AES128 的加密解密操作。

实验目的:

\begin{enumerate}
    \item 熟练掌握AES加密算法的理论
    \item 学习 rust 语言的基本用法
\end{enumerate}

实验平台:

\begin{enumerate}
    \item rust 语言
    \item Arch Linux (不依赖具体操作系统,rust 亦可在 windows/macOS 上使用)
\end{enumerate}

\chapter{实验内容}

\section{字节替代变换和逆字节替代变换}

    根据16×16的字节替代矩阵和逆替代矩阵,对于每个字节,将其前4bit作为横坐标,后4bit作为纵坐标,使用下方替代矩阵对应位置的字节进行替代。完成对一个长文本的字节替代变换和逆字节替代变换,对于不足16字符的输入需要补位,为了方便之后的加解密,16字节的文本应转换为4×4的字节矩阵。

{\heiti 解:}

因为 AES128 是作用于块大小为 128 bits 的块加密算法。
所以本题首先需要实现将明文消息分块,并且对于最后一块不足 16 字节时进行补位。
再对每一块进行字节替代,再进行逆字节替代。
最后进行逆补位得到原文。

\subsection{补位}

  \begin{minted}[texcomments,tabsize=2,fontsize=\normalsize,style=friendly,bgcolor=friendlybg]{rust}
fn pad(states: &mut Vec<u8>) -> () {
  let mut pad_size = states.len() % BLOCK_SIZE;
  if pad_size != 0 {
      pad_size = BLOCK_SIZE - pad_size;
  }
  states.extend(vec![PAD_BYTE; pad_size]);
}
  \end{minted}

其中 \texttt{PAD\_BYTE} 为 $0$,即用 $0$ 来补位,\texttt{BLOCK\_SIZE} 为 $16$ 字节,即 $128$ bits。

\subsection{字节替代}

  \begin{minted}[texcomments,tabsize=2,fontsize=\footnotesize,style=friendly,bgcolor=friendlybg]{rust}
fn sub_bytes(states: &mut Vec<u8>) -> () {
  for idx in 0..states.len() {
      let b = states[idx];
      states[idx] = SBOX[(b >> 4) as usize][(b & 0x0f) as usize];
  }
}
  \end{minted}

其中 SBOX 是提前给定的,SBOX 就是一个二维数组,第一维就是字节的高 4 位,第二维就是字节的低 4 位,具体的 SBOX 定义见源代码/实验书。

\subsection{逆字节替代}

  \begin{minted}[texcomments,tabsize=2,fontsize=\footnotesize,style=friendly,bgcolor=friendlybg]{rust}
fn inv_sub_bytes(states: &mut Vec<u8>) -> () {
    for idx in 0..states.len() {
        let b = states[idx];
        states[idx] = INV_SBOX[(b >> 4) as usize][(b & 0x0f) as usize];
    }
}
  \end{minted}

INV\_SBOX 就是根据 SBOX 对应的逆替换表。

\subsection{逆补位}

  \begin{minted}[texcomments,tabsize=2,fontsize=\footnotesize,style=friendly,bgcolor=friendlybg]{rust}
fn unpad(states: &mut Vec<u8>) -> () {
    let blocks = states.chunks_mut(BLOCK_SIZE);
    let last_block: &mut [u8] = blocks.last().unwrap();
    let padding_len = BLOCK_SIZE - match last_block.iter().position(
        |&x| x == PAD_BYTE) {
        Some(pos) => pos,
        None => BLOCK_SIZE,
    };
    for _ in 0..padding_len {
        states.remove(states.len() - 1);
    }
}
  \end{minted}

逆补位稍微复杂一点,首先需要在逆变换后的补位长度,然后将补位部分截断即可。

\subsection{运行结果}

\begin{figure}[h]
\centering%
\includegraphics[width=.4\linewidth]{aes_t1.png}
  \caption{字节替换运行结果}
  \label{fig:t1}
\end{figure}

本题使用字符串'abcdefghijklmn'作为输入,完成字符串的补位、转码、字节替代、逆字节替代、转码、去补位,输出每一步的结果。
运行结果如图~\ref{fig:t1}所示。

\section{行移位变换和逆行移位变换,列混合和逆列混合}

编写对4×4字节矩阵的行移位变换和逆行移位变换代码。编写对4×4字节矩阵的列混合和逆列混合代码,列混合需要使用x乘法 (xtime)。

\subsection{行移位变换}

行移位就是将每一块(block)用状态矩阵(state)表示,然后对每一行做一些简单的循环左移运算。
具体的,第一行不移位,第二行循环左移一位,第三行循环左移两位,第四行循环左移三位。

  \begin{minted}[texcomments,tabsize=2,fontsize=\footnotesize,style=friendly,bgcolor=friendlybg]{rust}
fn shift_rows(states: &mut Vec<u8>) -> () {
    let blocks = states.chunks_mut(BLOCK_SIZE);
    for state in blocks {
        let mut temp: u8;
        // row 1
        temp = state[1];
        state[1] = state[5];
        state[5] = state[9];
        state[9] = state[13];
        state[13] = temp;
        // row 2
        temp = state[2];
        state[2] = state[10];
        state[10] = temp;
        temp = state[6];
        state[6] = state[14];
        state[14] = temp;
        // row 3
        temp = state[15];
        state[15] = state[11];
        state[11] = state[7];
        state[7] = state[3];
        state[3] = temp;
    }
}
  \end{minted}

\subsection{列混合}

在列混合中,状态矩阵中的每一个字节都可以看作是 $GF(2^8)$ 有限域上的多项式,且最高项次数不超过 $7$。
在 $GF(2^8)$ 上乘以另外一个模多项式 $c(x)$ 再取模可以写作一个矩阵运算(由 $GF(2^8)$ 上定义的乘法实现的)。
对于 输入多项式 $a(x)$,乘以模多项式 $c(x) = '03'x^3+'01'x^2+'01'x+'02'$ 可以表示为:
\begin{align}
  \begin{split}
    b(x) &= c(x) \otimes a(x) \\
    &= \begin{bmatrix}
      02 & 03 & 01 & 01 \\
      01 & 02 & 03 & 01 \\
      01 & 01 & 02 & 03 \\
      03 & 01 & 01 & 02
    \end{bmatrix} \begin{bmatrix}
      a_0 \\
      a_1 \\
      a_2 \\
      a_3
    \end{bmatrix}
  \end{split}
\end{align}

注意 $GF(2^8)$ 上的加法就是异或运算。
同时,$GF(2^8)$ 上一个多项式乘以任意多项式(任意常数)都可以归约为乘以常数 '02' (也就是多项式 $x$,也叫做 x 乘法)和 '01' (那就是本身啦)。
例如,上面矩阵的结果中,有以下推导:

\begin{align*}
    b_0 &= '02' \cdot a_0 + '03' \cdot a_1 + '01' \cdot a_2 + '01' \cdot a_3 \\
    '02' \cdot a_0 + a_1 \cdot ('01' + '02') + a_2 + a_3 \\
    &= (a_0 + a_1) \cdot '02' + a_1 + a_2 + a_3 \\
    &= (a_0 + a_1) \cdot '02' + a_0 + a_0 + a_1 + a_2 + a_3
\end{align*}
其中 $GF(2^8)$ 上的加法运算就是二进制的逐位异或(xor),这里面利用了一些异或的性质就不再展开了,比较容易自己推导出来。

x 乘法(xtime)的实现代码如下:

  \begin{minted}[texcomments,tabsize=2,fontsize=\footnotesize,style=friendly,bgcolor=friendlybg]{rust}
fn xtime(b: u8) -> u8 {
    let c;
    if (b & 0x80) != 0 {
        c = (b << 1) ^ 0x1b;
    } else {
        c = b << 1;
    };
    c
}
  \end{minted}

代码看起来是简单,不过推导起来有不少步骤。
首先 $GF(2^8)$ 上的数,就是对应一个多项式,例如 $'1B' (0b0001\_1011)$ 就对应多项式 $x^4 + x^3 + x + 1$。
两个 $GF(2^8)$ 上的数相乘,就是对应他们的多项式在 $\mathbb{R}$ 上的普通乘法。
比如 x 乘法就是将数乘以多项式 $x$ (也就是 '02'),例如 $'1B' \cdot '02' = (x^4 + x^3 + x + 1) x = x^5 + x^4 + x^2 + x = 0b0011\_0110 = '54'$,其实就是二进制左移一位.
还有就是取模运算也不用于 $\mathbb{R}$ 上的取模,$GF(2^8)$ 上的取模笼统地来说就是用异或实现的除法。
举个例子,计算 $'B3' \cdot '02' \;(\bmod\; '11B') = 0b1011\_0011 \ll 2 \;(\bmod\; 0b1\_0001_1011) = 0b1\_0110\_0110 \;(\bmod 0b1\_0001\_1011)$ 过程如下:

{\tt
{\noindent-------------}\\
\phantom{0 }1 0110 0110 (mod) 1 0001 1011 \\
\textasciicircum\phantom{ }1 0001 1011 \\
------------- \\
\phantom{0 }0 0111 1101
}

其中上面的 $\textasciicircum$ 就是异或运算。
又因为 $GF(2^8)$ 最高阶就是 $x^7$ (二进制的最高位), 而模多项式 $'11B'$ 对应的最高阶是 $x^8$,也就是多一位。
所以 x 乘法的被乘出只有两种可能:(1)最高位是 $0$,则左移一位也不会超过 $'11B'$,所以取模后仍未原数,(2)最高位是 $1$,则左移一位后可能会大于模多项式,这时候我们需要使用上述取模的过程,与模多项式进行一次异或运算。
综上,这个代码逻辑就很好理解了。

有了 x 乘法,我们就可以得出乘以任意多项式的算法了,下面的代码就是列混淆的最终实现。

  \begin{minted}[texcomments,tabsize=2,fontsize=\footnotesize,style=friendly,bgcolor=friendlybg]{rust}
fn mix_columns(states: &mut Vec<u8>) {
    let blocks = states.chunks_mut(BLOCK_SIZE);
    for state in blocks {
        let columns = state.chunks_mut(ROW_COUNT);
        for column in columns {
            let tmp = column[0] ^ column[1] ^ column[2] ^ column[3];
            let bak_c0 = column[0];
            column[0] = xtime(column[0] ^ column[1]) ^ column[0] ^ tmp;
            column[1] = xtime(column[1] ^ column[2]) ^ column[1] ^ tmp;
            column[2] = xtime(column[2] ^ column[3]) ^ column[2] ^ tmp;
            column[3] = xtime(column[3] ^ bak_c0) ^ column[3] ^ tmp;
        }
    }
}
  \end{minted}


\subsection{逆列混合}

同样的,类似于列混淆,逆列混淆也是一个矩阵运算,只不过是矩阵是另外一个,运算过程为:
\begin{align}
  \begin{split}
    b(x) &= c(x) \otimes a(x) \\
    &= \begin{bmatrix}
0e & 0b & 0d & 09 \\
09 & 0e & 0b & 0d \\
0d & 09 & 0e & 0b \\
0b & 0d & 09 & 0e \\
    \end{bmatrix} \begin{bmatrix}
      a_0 \\
      a_1 \\
      a_2 \\
      a_3
    \end{bmatrix}
  \end{split}
\end{align}

从矩阵可以看出来,如果用 x 乘法来实现上面的多项式乘法比列混淆要复杂很多,因为都是乘以 $0e$, $0d$ 这样的比较大的数,需要的 x 乘法也自然会多一些。
AES 专利论文中有相应的优化算法,不过我在代理里面仍然使用的是最 naive 的实现方法。

  \begin{minted}[texcomments,tabsize=2,fontsize=\footnotesize,style=friendly,bgcolor=friendlybg]{rust}
fn inv_mix_columns(states: &mut Vec<u8>) {
    let blocks = states.chunks_mut(BLOCK_SIZE);
    for state in blocks {
        let columns = state.chunks_mut(ROW_COUNT);
        for column in columns {
            let mut t = column[0] ^ column[1] ^ column[2] ^ column[3];
            let u = xtime(xtime(column[0] ^ column[2]));
            let v = xtime(xtime(column[1] ^ column[3]));
            let bak_c0 = column[0];
            column[0] = t ^ column[0] ^ xtime(column[0] ^ column[1]);
            column[1] = t ^ column[1] ^ xtime(column[1] ^ column[2]);
            column[2] = t ^ column[2] ^ xtime(column[2] ^ column[3]);
            column[3] = t ^ column[3] ^ xtime(column[3] ^ bak_c0);
            t = xtime(u ^ v);
            column[0] ^= t ^ u;
            column[1] ^= t ^ v;
            column[2] ^= t ^ u;
            column[3] ^= t ^ v;
        }
    }
}
  \end{minted}

\subsection{逆行移位变换}

逆行移位变换就是行移位运算的逆过程,就是将之前的循环左移变成循右移。

  \begin{minted}[texcomments,tabsize=2,fontsize=\footnotesize,style=friendly,bgcolor=friendlybg]{rust}
fn inv_shift_rows(states: &mut Vec<u8>) -> () {
    let blocks = states.chunks_mut(BLOCK_SIZE);
    for state in blocks {
        let mut temp: u8;
        temp = state[13];
        state[13] = state[9];
        state[9] = state[5];
        state[5] = state[1];
        state[1] = temp;
        temp = state[14];
        state[14] = state[6];
        state[6] = temp;
        temp = state[10];
        state[10] = state[2];
        state[2] = temp;
        temp = state[3];
        state[3] = state[7];
        state[7] = state[11];
        state[11] = state[15];
        state[15] = temp;
    }
}
  \end{minted}

\subsection{运行结果}

本题使用字符串'abcdefghijklmnop'作为验证输入,将其转换为字节矩阵后对四个变换进行验证,记录其输出。
行移位变换和列混合运行结果如图~\ref{fig:t2}所示。

\begin{figure}[h]
\centering%
\includegraphics[width=.3\linewidth]{aes_t2.png}
  \caption{行移位变换和列混合运行结果}
  \label{fig:t2}
\end{figure}

\section{轮密钥生成}

\subsection{轮密钥生成算法}

为了把一个 128 bits 的密钥扩展到 AES128 的 10 轮迭代中,我们需要使用密钥扩展算法, 把原始密钥再扩展出 10 个轮密钥(round key)。
其实 AES 的密钥扩展比较简单,仍然使用的是 RotByte 字节循环右移(4个字节构成一组,或者说 state 中的一列)和 使用 SBOX 的 SubByte 字节替换来实现的。
在最后还要再加上(异或) RCON 中规定的 $GF(2^8)$ 多项式。

具体来说,每一轮将生成 16 字节长度的密钥,每一个轮密钥的前 4 个字节为前面四个字节(也就是上一个轮密钥的最后 4 个字节)先 RotByte 再 SubByte ,并且最后还要对第一个字节加上(异或)\texttt{Rcon[i]} (i 表示当前为第 i 轮)。
\texttt{Rcon[i]} 直接可以用题目给的那个数组,具体的计算公式见 AES 专利论文~\cite{aes}。 

代码实现如下,

  \begin{minted}[texcomments,tabsize=2,fontsize=\footnotesize,style=friendly,bgcolor=friendlybg]{rust}
fn key_schedule(cipher_key: &[u8; BLOCK_SIZE]) -> [u8; (ROUND + 1) * BLOCK_SIZE] {
    let mut round_keys = [0u8; (ROUND + 1) * BLOCK_SIZE];
    for i in 0..BLOCK_SIZE { round_keys[i] = cipher_key[i]; }
    for i in 1..(ROUND + 1) {
        // last 4 bytes (one word), i.e., last column
        let mut temp_offset: usize = i * 16 - 4;
        // last round (16 bytes), i.e., last round key
        let mut last_round: usize = (i - 1) * 16;
        // RotByte is cyclic right shift, e.g., (a, b, c, d) => (b, c, d, a)
        round_keys[i * 16] = sub_byte(round_keys[temp_offset + 1]) ^ RCON[i] ^ round_keys[
            last_round + 1];
        round_keys[i * 16 + 1] = sub_byte(round_keys[temp_offset + 2]) ^ round_keys[
            last_round + 2];
        round_keys[i * 16 + 2] = sub_byte(round_keys[temp_offset + 3]) ^ round_keys[
            last_round + 3];
        round_keys[i * 16 + 3] = sub_byte(round_keys[temp_offset]) ^ round_keys[
            last_round];
        for j in 1..4 {
            temp_offset += 4;
            last_round += 4;
            round_keys[i * 16 + j * 4] = round_keys[last_round] ^ round_keys[temp_offset];
            round_keys[i * 16 + j * 4 + 1] = round_keys[last_round + 1] ^ round_keys[
                temp_offset + 1];
            round_keys[i * 16 + j * 4 + 2] = round_keys[last_round + 2] ^ round_keys[
                temp_offset + 2];
            round_keys[i * 16 + j * 4 + 3] = round_keys[last_round + 3] ^ round_keys[
                temp_offset + 3];
        }
    }
    round_keys
}
\end{minted}

\subsection{运行结果}

使用原始密钥'abcdefghijklmnop'生成总计11组扩展密钥,将用于之后的加解密。初始密钥K转换为字节矩阵后的4列为 $W\_0 \sim W\_3$,后续的 $W\_4 \sim W\_{43}$ 使用递归计算得到。
计算结果如图~\ref{fig:t3}所示。

\begin{figure}[h]
\centering%
\includegraphics[width=\linewidth]{aes_t3.png}
  \caption{轮密钥生成运行结果}
  \label{fig:t3}
\end{figure}

\section{加密机和解密机}

\subsection{代码实现}

加密和解密的逻辑就比较简单了,就是把之前的那些功能整合到一起。
加密过程包含字节填充,10 轮加密,其中前 9 轮都是一模一样的,最后一轮差别在于没有列混淆过程。
每一轮加密依次包括字节替换,行移位,列混淆,轮密钥加。
解密过程就是加密过程的逆过程,需要注意的是,轮密钥的顺序也是反过来的。

具体的加密机代码如下:

  \begin{minted}[texcomments,tabsize=2,fontsize=\footnotesize,style=friendly,bgcolor=friendlybg]{rust}
fn aes128_encrypt(plaintext: &str, cipher_key: &str) -> Vec<u8> {
    let mut states = plaintext.to_string().into_bytes();
    assert!(cipher_key.is_ascii());
    assert_eq!(cipher_key.len(), BLOCK_SIZE);
    let cipher: [u8; BLOCK_SIZE] = <[u8; BLOCK_SIZE]>::try_from(
        cipher_key.to_string().into_bytes()).unwrap();
    let round_keys = key_schedule(&cipher);
    pad(&mut states);
    // first AddRoundKey
    add_round_key(&mut states, &round_keys[0..BLOCK_SIZE]);
    // 9 rounds
    for i in 1..ROUND {
        sub_bytes(&mut states);
        shift_rows(&mut states);
        mix_columns(&mut states);
        add_round_key(&mut states, &round_keys[i * BLOCK_SIZE..(i + 1) * BLOCK_SIZE]);
    }
    // last round
    sub_bytes(&mut states);
    shift_rows(&mut states);
    add_round_key(&mut states, &round_keys[10 * BLOCK_SIZE..11 * BLOCK_SIZE]);
    states
}
\end{minted}

具体的解密机代码如下:

  \begin{minted}[texcomments,tabsize=2,fontsize=\footnotesize,style=friendly,bgcolor=friendlybg]{rust}
fn aes128_decrypt(cipher: &Vec<u8>, cipher_key: &str) -> String {
    let mut states = cipher.clone();
    assert!(cipher_key.is_ascii());
    assert_eq!(cipher_key.len(), BLOCK_SIZE);
    let key_array: [u8; BLOCK_SIZE] = <[u8; BLOCK_SIZE]>::try_from(
        cipher_key.to_string().into_bytes()).unwrap();
    let round_keys = key_schedule(&key_array);
    // from the last round to the first round
    let mut r_offset = BLOCK_SIZE * 10;
    // first AddRoundKey
    add_round_key(&mut states, &round_keys[r_offset..r_offset + BLOCK_SIZE]);
    inv_shift_rows(&mut states);
    inv_sub_bytes(&mut states);
    // 9 rounds
    for _ in 1..ROUND {
        r_offset -= BLOCK_SIZE;
        add_round_key(&mut states, &round_keys[r_offset..r_offset + BLOCK_SIZE]);
        inv_mix_columns(&mut states);
        inv_shift_rows(&mut states);
        inv_sub_bytes(&mut states);
    }
    // last round
    r_offset -= BLOCK_SIZE;
    add_round_key(&mut states, &round_keys[r_offset..r_offset + BLOCK_SIZE]);
    unpad(&mut states);
    let result_text = String::from_utf8(states.to_vec())
        .unwrap();
    result_text
}
\end{minted}
\subsection{运行结果}

整合轮函数,完成AES加密和解密,原始密钥使用'abcdefghijklmnop'
明文为 'Cryptography and Network Security; 2020214245; 肖文韬 (Wentao Xiao) 🎉🚀' ,使用自己的学号和姓名拼音替代对应位置字符串。
最终加密机和解密机的运行结果如图~\ref{fig:t4}所示。

\begin{figure}[h]
\centering%
\includegraphics[width=\linewidth]{aes_t4.png}
  \caption{加密机和解密机运行结果}
  \label{fig:t4}
\end{figure}

% 其他部分
\backmatter

% 参考文献
\bibliography{ref/refs}  % 参考文献使用 BibTeX 编译

\end{document}
