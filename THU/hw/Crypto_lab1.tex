% !TeX encoding = UTF-8
% !TeX program = xelatex
% !TeX spellcheck = en_US

\documentclass[degree=project,degree-type=project,cjk-font=noto]{thuthesis}
\usepackage{mathtools}
\usepackage{tikz}
\usetikzlibrary{shapes,arrows}
\usepackage[autosize]{dot2texi}
% Syntax Highlighting in LaTeX, need pygments
% Must build with xelatex -shell-escape -enable-8bit-chars.
\usepackage{minted}
% https://tex.stackexchange.com/a/112573
\usepackage{tcolorbox}
\usepackage{etoolbox}
\BeforeBeginEnvironment{minted}{\begin{tcolorbox}}%
\AfterEndEnvironment{minted}{\end{tcolorbox}}%
% color for minted
\definecolor{friendlybg}{HTML}{f0f0f0}


% 论文基本配置,加载宏包等全局配置
\thusetup{
    output = electronic,
    title  = {实验一},
    author  = {肖文韬},
    studentid = {2020214245},
    major = {电子信息(计算机技术)},
    email = {xwt20@mails.tsinghua.edu.cn},
    course = {密码学与网络安全},
    include-spine = false,
}


\usepackage{float}
\usepackage[sort]{natbib}
\bibliographystyle{thuthesis-numeric}
\graphicspath{{figures/}}


\setlist[enumerate,1]{label=\arabic*.}
\setlist[enumerate,2]{label=(\alph*)}
\setlist[enumerate,3]{label=\roman*.}
\setlist[enumerate,4]{label=\greek*}


\begin{document}

% 封面
\maketitle

\frontmatter
% \input{data/abstract}

% 目录
% \tableofcontents

% 插图和附表清单
% \listoffiguresandtables
% \listoffigures           % 插图清单

% 正文部分
\mainmatter

\chapter{实验介绍}
AES加密算法是迭代型分组密码算法,涉及4种操作:字节替代、行移位、列混合和轮密钥加。
本次实验中密钥长度和分组长度都为128比特,加密轮数为10轮。
实验使用 Rust 语言实现 AES128 的加密解密操作。

实验目的:

\begin{enumerate}
    \item 熟练掌握AES加密算法的理论
    \item 学习 rust 语言的基本用法
\end{enumerate}

实验平台:

\begin{enumerate}
    \item rust 语言
    \item Arch Linux (不依赖具体操作系统,rust 亦可在 windows/macOS 上使用)
\end{enumerate}

\chapter{实验内容}

\section{字节替代变换和逆字节替代变换}

    根据16×16的字节替代矩阵和逆替代矩阵,对于每个字节,将其前4bit作为横坐标,后4bit作为纵坐标,使用下方替代矩阵对应位置的字节进行替代。完成对一个长文本的字节替代变换和逆字节替代变换,对于不足16字符的输入需要补位,为了方便之后的加解密,16字节的文本应转换为4×4的字节矩阵。

本题使用字符串'abcdefghijklmn'作为输入,完成字符串的补位、转码、字节替代、逆字节替代、转码、去补位,输出每一步的结果。

\section{行移位变换和逆行移位变换,列混合和逆列混合}

    编写对4×4字节矩阵的行移位变换和逆行移位变换代码。编写对4×4字节矩阵的列混合和逆列混合代码,列混合需要使用x乘法。

本题使用字符串'abcdefghijklmnop'作为验证输入,将其转换为字节矩阵后对四个变换进行验证,记录其输出。

\section{轮密钥生成}

    使用原始密钥'abcdefghijklmnop'生成总计11组扩展密钥,将用于之后的加解密。初始密钥K转换为字节矩阵后的4列为 $W_0 \sim W_3$,后续的 $W_4 \sim W_{43}$ 使用递归计算得到。

\end{enumerate}

\end{document}
