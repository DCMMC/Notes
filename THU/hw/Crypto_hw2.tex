% !TeX encoding = UTF-8
% !TeX program = xelatex
% !TeX spellcheck = en_US

\documentclass[degree=project,degree-type=project,cjk-font=noto]{thuthesis}
\usepackage{mathtools}
\usepackage{tikz}
\usetikzlibrary{shapes,arrows}
\usepackage[autosize]{dot2texi}
% Syntax Highlighting in LaTeX, need pygments
% Must build with xelatex -shell-escape -enable-8bit-chars.
\usepackage{minted}
% https://tex.stackexchange.com/a/112573
\usepackage{tcolorbox}
\usepackage{etoolbox}
\BeforeBeginEnvironment{minted}{\begin{tcolorbox}}%
\AfterEndEnvironment{minted}{\end{tcolorbox}}%
% color for minted
\definecolor{friendlybg}{HTML}{f0f0f0}


% 论文基本配置,加载宏包等全局配置
\thusetup{
    output = electronic,
    title  = {作业二},
    author  = {肖文韬},
    studentid = {2020214245},
    major = {电子信息(计算机技术)},
    email = {xwt20@mails.tsinghua.edu.cn},
    course = {密码学与网络安全},
    include-spine = false,
}


\usepackage{float}
\usepackage[sort]{natbib}
\bibliographystyle{thuthesis-numeric}
\graphicspath{{figures/}}


\setlist[enumerate,1]{label=\arabic*.}
\setlist[enumerate,2]{label=(\alph*)}
\setlist[enumerate,3]{label=\roman*.}
\setlist[enumerate,4]{label=\greek*}


\begin{document}

% 封面
\maketitle

\frontmatter
% \input{data/abstract}

% 目录
% \tableofcontents

% 插图和附表清单
% \listoffiguresandtables
% \listoffigures           % 插图清单

% 正文部分
\mainmatter

\chapter{作业内容}

\begin{enumerate}
  \setlength{\itemsep}{3\parskip}
  \item 考虑一个密码体系 $M = \{a, b, c\}, K = \{k_1, k_2, k_3\}$ 和 $C = \{1, 2, 3, 4\}$, 将明文 M 使用密钥 K 加密为密文 C。假设加密矩阵如下表。
  \begin{table}[htp]
  	\centering
  	\begin{tabular}{|c|c|c|c|}
  		\hline
  		& a & b & c \\\hline
  		$k_1$ & 2 & 3 & 4 \\\hline
  		$k_2$ & 3 & 4 & 1 \\\hline
  		$k_3$ & 1 & 2 & 3 \\\hline
	\end{tabular}
  \end{table}
  \newline
  已知密钥概率分布 $p(k_3) = 1/2, p(k_2) = p(k_1) = 1/4$, 且明文概率分布 $p(a) = 1/3, p(b) = 8/15, p(c)  = 2/15$。请计算 $H(M), H(K), H(C), H(M|C), H(K|C)$。
  \newline
  {\heiti 解:}
  \begin{align}
  	H(M) &= \sum_i p(M_i) I(M_i) = -\sum_i p(M_i) \log p(M_i) \approx 1.3996 \\
  	H(K) &\approx 1.5 \\
  	H(C) &\approx 1.9408 \quad (p(C) = [0.2, 0.35, 0.2833, 0.1667]) \\
  	H(M|C) &= H(MC) - H(C) \nonumber \\
          &= -\sum_i \sum_j p(M_i, C_j) \log p(M_i, C_j) \approx 2.8996 \\
  	H(K|C) &= H(KC) - H(C) \approx 2.8996
  \end{align}

\item 计算英文字母的凯撒密码的唯一解距离。
\newline
{\heiti 解:}
\newline
\begin{equation}
N = \frac{H(K)}{D} = - \frac{\log\left(\frac{1}{26}\right)}{3.2} \approx 2
\end{equation}
\newline

\item 计算重复周期为 6 的维吉尼亚密码的唯一解距离。
\newline
{\heiti 解:}
\newline
\begin{equation}
N = \frac{H(K)}{D} = - \frac{\log \left(\frac{1}{26^6}\right)}{3.2} \approx 9
\end{equation}
\newline

\item 某次 AES 加密的轮函数过程中,字节替代的结果为:
\begin{equation}
  A = \begin{bmatrix}
  87 & F2 & 4D & 97 \\
  EC & 6E & 4C & 90 \\
  4A & C3 & 46 & E7 \\
  8C & D8 & 95 & A6
  \end{bmatrix}
\end{equation}
{\heiti 解:}
\newline
求这个矩阵经过行移位变换后的结果,以及经过列混淆后第三行第一列的值。

\begin{figure}[!htp]
\centering%
\includegraphics[width=.7\linewidth]{aes.png}
  \caption{移位变换和列混淆的具体过程}
  \label{fig:aes}
\end{figure}

移位变换和列混淆的具体过程如图~\ref{fig:aes}所示。
移位表换后的结果为 $B$.

\begin{equation}
  B = \begin{bmatrix}
  87 & F2 & 4D & 97 \\
  6E & 4C & 90 & EC \\
  46 & E7 & 4A & C3 \\
  A6 & 8C & D8 & 95
  \end{bmatrix}
\end{equation}

列混淆后的结果为 $C$.

\begin{equation}
  C = \begin{bmatrix}
  47 & 40 & A3 & 4C \\
  37 & D4 & 70 & 9F \\
  94 & E4 & 3A & 42 \\
  ED & A5 & A6 & BC
  \end{bmatrix}
\end{equation}

\item 思考题:为何 AES 加密算法的最后一轮与前 9 轮不同?

{\heiti 解:}
\newline
AES算法在处理的轮数上只有最后一轮与前面 9 轮不同之处在于最后一轮少了列混淆处理。
理由:
\newline\newline
在正常的 AES 轮中,列混淆会在轮密钥加操作之前进行。
不过,也可以调换这些操作的顺序。先执行轮密钥加操作,再执行列混淆操作,稍加修改,就可以收到同样的结果。
因此,可以认为最后的列混淆不会增加任何安全性,因为它是一个不加键、可逆的操作,可以使其成为最后一轮的最后一步。
\newline\newline
然而理论上我们可以进行攻击。考虑到一个AES变体,其中列混淆在最后一轮加密中执行。
为了攻击解密函数,攻击者可能会交换列混淆和轮密钥加的顺序,这样他就可以直接撤销列混淆。
现在假设他能够(以某种方式)恢复轮密钥加中使用的轮密钥的一些信息。
因为他交换了操作,所以他恢复的实际上并不是键表(Key schedule)吐出的轮密钥的信息,而是应用了逆列混淆的轮密钥的信息。
\end{enumerate}

\end{document}
