% !TeX encoding = UTF-8
% !TeX program = xelatex
% !TeX spellcheck = en_US

\documentclass[degree=project,degree-type=project,cjk-font=noto]{thuthesis}
% 著者-出版年制
\usepackage[sort]{natbib}
\bibliographystyle{thuthesis-author-year}
\usepackage{mathtools}
\usepackage{tikz}
\usetikzlibrary{shapes,arrows}
\usepackage[autosize]{dot2texi}
% Syntax Highlighting in LaTeX, need pygments
% Must build with xelatex -shell-escape -enable-8bit-chars.
\usepackage{minted}
% https://tex.stackexchange.com/a/112573
\usepackage{tcolorbox}
\usepackage{etoolbox}
\BeforeBeginEnvironment{minted}{\begin{tcolorbox}}%
\AfterEndEnvironment{minted}{\end{tcolorbox}}%
% color for minted
\definecolor{friendlybg}{HTML}{f0f0f0}


% 论文基本配置,加载宏包等全局配置
\thusetup{
    output = electronic,
    title  = {数据隐私在营销中的作用},
    author  = {肖文韬},
    studentid = {2020214245},
    major = {电子信息(计算机技术)},
    email = {xwt20@mails.tsinghua.edu.cn},
    course = {数据伦理},
    include-spine = false,
}

\usepackage{float}
\graphicspath{{figures/}}


\begin{document}

% 封面
\maketitle

\frontmatter
\begin{abstract}
  本文通过阅读老师给出的英文文献,总结了市场营销及相关学科中隐私有关学术研究的现状。
  正文部分主要从下面三个方面进行展开:

  \begin{enumerate}
    \item 隐私在社会中的作用
    \item 隐私心理学
    \item 隐私经济学
  \end{enumerate}



  \thusetup{
    keywords = {数据隐私, 营销},
	}
\end{abstract}

% 目录
\tableofcontents

% 插图和附表清单
% \listoffiguresandtables
% \listoffigures           % 插图清单

% 正文部分
\mainmatter

\chapter{引言}

广泛获取消费者个人信息的影响是多方面的,包括容易受到欺诈、隐私侵犯、不受欢迎的营销传播,以及目标明确、扰乱日常活动节奏的营销传播。


\chapter{隐私在社会中的作用}

\chapter{隐私心理学}

\chapter{隐私经济学}

% 其他部分
\backmatter

% 参考文献
\bibliography{ref/refs}  % 参考文献使用 BibTeX 编译

% 附录
\appendix

\end{document}
