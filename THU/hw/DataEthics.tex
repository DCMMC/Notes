% !TeX encoding = UTF-8
% !TeX program = xelatex
% !TeX spellcheck = en_US

\documentclass[degree=project,degree-type=project,cjk-font=windows]{thuthesis}
% 著者-出版年制
\usepackage[sort]{natbib}
\bibliographystyle{thuthesis-author-year}
\usepackage{mathtools}
\usepackage{tikz}
\usetikzlibrary{shapes,arrows}
\usepackage[autosize]{dot2texi}
% Syntax Highlighting in LaTeX, need pygments
% Must build with xelatex -shell-escape -enable-8bit-chars.
\usepackage{minted}
% https://tex.stackexchange.com/a/112573
\usepackage{tcolorbox}
\usepackage{etoolbox}
\BeforeBeginEnvironment{minted}{\begin{tcolorbox}}%
\AfterEndEnvironment{minted}{\end{tcolorbox}}%
% color for minted
\definecolor{friendlybg}{HTML}{f0f0f0}


% 论文基本配置,加载宏包等全局配置
\thusetup{
    output = electronic,
    title  = {数据隐私在市场营销中的作用},
    author  = {肖文韬},
    studentid = {2020214245},
    major = {电子信息(计算机技术)},
    email = {xwt20@mails.tsinghua.edu.cn},
    course = {数据伦理},
    include-spine = false,
}

\usepackage{float}
\graphicspath{{figures/}}


\begin{document}

% 封面
\maketitle

\frontmatter
\begin{abstract}
  本文通过阅读老师给出的英文文献,总结了市场营销及相关学科中隐私有关学术研究的现状。
  研究方法部分主要从下面三个方面进行展开介绍:

  \begin{enumerate}
    \item 隐私在社会中的作用
    \item 隐私心理学
    \item 隐私经济学
  \end{enumerate}

  最后我们将总结全文并讨论论文亮点与不足。

  \thusetup{
    keywords = {数据隐私, 市场营销},
	}
\end{abstract}

% 目录
\tableofcontents

% 插图和附表清单
% \listoffiguresandtables
% \listoffigures           % 插图清单

% 正文部分
\mainmatter

\chapter{引言}

广泛获取消费者个人信息的影响是多方面的,包括容易受到欺诈、隐私侵犯、不受欢迎的营销传播,以及目标明确、扰乱日常活动节奏的营销传播。
但更多的时候,个人信的息利用给消费者带来的好处被广泛宣传。
对消费者数据的精密使用,可以提供个性化的产品和建议、价格折扣、免费服务以及更相关的营销传播和媒体内容。
理论上,营销人员可以将更多的利益传递给消费者,因为他们能够利用更好的信息更有效地运作。
\cite{data_privacy}的论文通过大量关于隐私和消费者数据使用的营销文献(以及信息系统、法律、伦理学和其他学科),以及在这一领域仍有待理解的东西。
这些观察普遍表明,使用消费者数据和分析的营销实践比营销的学术研究的发展速度更快。
具体来说,论文认为,在现实中,有意义的问题已经从消费者是否愿意披露他们的私人信息转移到了现在消费者如何反应,因为他们的私人信息被广泛地访问并提供给众多营销人员和其他相关方。
数据并不是真正的问题,谁能看到和使用这些数据才是问题所在。现在把那个精灵放回瓶子里已经太晚了。
于是,\cite{data_privacy}的论文从隐私在社会中的作用、隐私心理学和隐私经济学三个角度进行阐述个人隐私在市场营销中的作用。
本文主要是阅读\cite{data_privacy}的论文后的评述。

首先,我们将在第\ref{chap2}介绍该论文的研究方法,也就是该论文的主要内容。
接着,该论文全文所要表达的内容将会在第\ref{chap3}章总结全文结论。
然后,我们在\ref{chap4}指出论文的亮点与不足。

\chapter{研究方法}
\label{chap2}

\section{隐私在社会中的作用}
隐私往往被铸成个人的 "权利",关于 "隐私权 "的讨论屡见不鲜。
美国的《权利法案》在一些修正案中嵌入了对隐私的保护(如第一、第三、第四和第五修正案)。有趣的是,在美国司法系统中,类似于隐私权的保护措施在各个法院都得到了遵守,并得到了高度重视。
然而,在没有专门的隐私权的情况下,联邦监管机构一直不愿意在各公司和政府实体之间实施隐私保护。
显然,我们需要进行更多的全球研究,研究国际人口中消费者和组织层面的隐私的相似性和差异性。我们对消费者、组织和法律/道德领域的隐私的理解大多限于美国和欧洲样本。尽管欧洲消费者表现出更高的隐私关注度,随后欧盟立法者也对这种关注进行了保护,但许多关于消费者隐私的正式知识都具有明显的西方味道。当然,一个重要的缺失空间涉及到东方更多集体主义文化社会的隐私问题,如印度和中国。

二十多年前,\cite{Bloom}提出了营销人员应该考虑的两个关键问题。

\begin{enumerate}
\item 是否应该允许公司在不了解或不同意的情况下获取和存储个人的信息?
\item 是否应该允许公司在不了解或不同意的情况下向其他方披露个人信息?
\end{enumerate}

虽然这一争论在文献中仍未解决,但营销人员还是像获得了个人信息的访问权和披露许可一样做出了回应,目前他们采集、存储和销售了大量的消费者数据。

其他道德观点显示出加强对隐私的理解的潜力。例如,\cite{Murphy}提出的权力-责任平衡模型主张社会权力和社会责任之间的联系,在这种情况下,一个强大的伙伴有义务保护和促进较弱的伙伴的感觉平等。应用于营销者与消费者的隐私问题,研究表明,当企业在信息交换中未能促进感觉上的平等时,消费者对权力不平衡的反应是消极的。在市场营销中,需要伦理理论的发展和分析来进一步分化当代数据隐私问题,以努力建议营销人员应该如何管理这些问题。例如,\cite{Ferrell}的道德决策模型可以帮助理解,将社会对信息使用的关注与公司机会和个人行为在负责任地对待消费者信息方面的作用联系起来。

\section{隐私心理学}
与支撑隐私在社会中的作用的基础不同的是,解释消费者隐私心理学的理论与之相差甚远。这种框架内聚性的差异可能源于这样一个事实,即社会视角主要关注的是营销人员应该做什么,而心理学视角则必须努力处理大量高度细微的方式,消费者可以解释隐私问题。
许多文献都集中在理解和测量作为消费者心理结构的隐私关注。隐私关注作为理解消费者对其信息隐私感受的最佳代表浮出水面。多年来,人们推进了对隐私关注的多种测量,并将其与各种驱动因素和结果一起进行了研究。史密斯等人提供了一个多维度的量表,包括(1)信息收集,(2)未经授权的二次使用(内部和外部),(3)不正当访问和(4)错误保护。他们的量表在2004年进行了更新,以更有针对性地考虑消费者在网络领域对信息隐私的关注。这种以测量为中心的工作,主要发表在信息系统期刊上,但在营销文献中的采用和使用比较有限。早期的市场营销中的隐私研究偏重于在各种情境下直接向消费者提问。例如,Sheehan和Hoy受1998年推进的FTC核心隐私原则的启发,开发了一些措施来研究隐私问题。具体来说,这些学者使用了一套工具,涵盖了(1)信息收集意识,(2)信息使用,(3)信息敏感性,(4)对实体的熟悉程度,以及(5)补偿。

由于人们普遍认为这是衡量消费者感受的最佳方法,因此,直到2010年左右,特定的隐私关注度作为重点结果和预测变量被广泛研究。作为预测变量,消费者对隐私问题的高度关注往往与一些常见的信息隐私结果相关联,包括披露信息的意愿和购买意向。然而,关于隐私担忧会降低信息披露和购买意向的研究结果比直观的结果要细微得多,表面上看可能会阻碍消费者与组织的互动。尽管早期的工作发现,更多的关注会导致消费者的负面反应增加,但其他研究发现,对隐私的关注是高度情境化的,并受到一些制约因素的约束,包括公平观念、隐私政策强度和公司信任。

作为一种结果衡量标准,老年和年轻消费者对隐私的关注似乎随着时间的推移而增加,尽管老年消费者对隐私的关注增加得更快。事实证明,随着个人和监管措施的加强,对隐私的关注也会减少。消费者感知到的控制也是一种中介机制,通过这种机制,隐私保护被证明是有效的。早期的工作发现,感知到的脆弱性,而不是感知到的控制,影响了隐私关注。最近,隐私关注是作为一个个体差异变量来衡量的,重点是了解不同消费者样本所报告的感觉到的控制的作用。在更复杂的信息隐私关系调查中,隐私关注被概念化为中介或调节条件。

\section{隐私经济学}

有一少部分研究机构研究了隐私的经济学,或者说,公司如何管理消费者隐私。研究敏感的、甚至有潜在争议的、对隐私敏感的公司行为会造成数据收集上的障碍;然而,了解公司对待消费者隐私的方式是至关重要的。在一项调查中,Milne和Bahl确实沟通了消费者和营销者的隐私偏好,他们发现了这些观点之间的协同作用和脱节。不足为奇的是,消费者比营销者更有可能选择自己和组织之间的封闭边界。确定了不情愿的消费者与接受的消费者的分组,前者对隐私相关技术的接受程度远低于后者。有趣的是,接受组对新技术的接受程度甚至比营销经理还高,后者报告说希望在消费者交流中更普遍地设置许可边界。这些研究结果表明,组织的隐私实践很可能至少在某种程度上偏离了客户的愿望和欲望,再次强调了进行更多企业隐私政策研究的必要性。

鉴于同时理解消费者和组织隐私问题所涉及的复杂性,公司方面的许多工作都采用经济模型技术来研究数据隐私结果。\cite{Rust}利用理论模型绘制了互联网隐私的消费者经济学。作者推导出一个不对隐私进行监管的自由市场体系,发现消费者隐私会受到侵蚀,以至于出现了隐私市场。消费者可以为一定的隐私付费,但随着隐私的不断侵蚀,企业提供的交换价值的质量会下降。同样,Conitzer等人建立了消费者重复购买的模型,具体说明了企业如何利用现有客户的信息在未来的购买中进行价格歧视。由该模型得出的命题表明,当购买者可以自由保持匿名时,企业在交换中获利最大,但同时也表明,当匿名成本有些高时,消费者获益最大。第三方隐私把关人也影响了消费者的价格。这两组作者都发现,这些把关人在与企业协商使匿名性免费时,对消费者不利。总的来说,这些研究表明,组织依靠一些基本的数据隐私保护来保证整个市场系统的平稳运行。

有一小部分研究对显著的信息安全漏洞,或者说是一种极端形式的隐私失败的公司结果进行了研究。信息安全失败和数据泄露事件正在上升,影响到世界各地不同行业越来越多的公司。尽管商业媒体估计公司因数据泄露造成的损失高达数百万美元,但研究大型隐私失败对企业绩效影响的学术研究仍然不够发达。研究隐私失效的研究主要是在信息系统文献中进行的,通常研究的是安全的技术增强如何减轻负面的违规影响,或者是易使公司发生违规的背景因素,如地理和行业。使这一知识体系更加复杂的是,对于隐私失败对公司的实际危害程度仍然存在一些疑问,使得公司没有什么理由加强隐私保护。

\chapter{结论}
\label{chap3}

该论文把握了市场营销及相关学科的隐私学术研究现状。
作者根据隐私在社会中的作用、隐私心理学和隐私经济学,对有关数据和信息隐私的理论观点和经验发现进行了研究。
强调了未来的研究主题,这些主题体现了多维度的方法,它融合了当代营销中的隐私问题中的许多相互关联的关注。由于内部和外部利益相关者受到数据隐私问题的多重和潜在的不可预见的影响,因此在这一领域的更多工作仍然是至关重要和必要的。

\chapter{论文亮点和不足}
\label{chap4}

该论文总结了市场营销及相关学科中数据隐私研究的现状。
正如隐私学者、从业者和监管者所承认的那样,消费者信息隐私的概念很难界定。
虽然已有的理论体系通过深入的见解提供了强有力的理解,但在某些方面,这种理论将我们对隐私的看法限制在消费者、组织、伦理和法律的孤岛上。
从营销领域庞大的隐私学术研究中提取的经验性发现和关系也呼应了这一观察,许多重大的研究进展发生在严格定义的空间中的狭窄关系中。
对此,该论文的亮点在于提醒我们采取必要的步骤,跨越这些边界扩大隐私领域。
通过综合这些领域的隐私,该论文倡导以一种整体的方式来思考消费者数据的组织使用,以及如何将其融入更大的社会图景。

不足之处在于,该论文将隐私作为战略的讨论只是提供了一些例子,还缺少将当代数据隐私问题中的许多消费者、组织、道德和法律问题融合在一起。
由于利益攸关方以多种可能无法预见的方式受到影响,因此在这一重要领域开展更多的工作仍然是至关重要和必要的。
另一个不足之处在于,论文的主要内容是综述,虽然抛出来了一个值得大家深思的问题,但是论文缺少解决思路。

% 其他部分
\backmatter

% 参考文献
\bibliography{ref/refs}  % 参考文献使用 BibTeX 编译

% 附录
\appendix

\end{document}
