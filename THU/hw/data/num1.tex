\begin{abstract}
	数值方法作为解决科学和工程问题的一个强有力工具,它在许多基础研究和工程应用上扮演着重要的角色。
	随着数学理论和计算机硬件的发展,各种数值方法的提出,新的数值方法或其新的应用使相关领域取得了重要进展。
	本文主要聚焦于数值方法在机器学习领域,尤其是在深度学习领域中的应用。
	机器学习算法常常需要用到大量的数值计算,因为计算复杂度,我们往往无法直接得出精确的符号解。
	同时,由于计算过程中存在的种种误差,不考虑误差得到的计算结果往往与正确的结果大相径庭。
	为了解决这些问题,数值方法的提出为我们带来了可靠的理论分析工具,以及将计算不可行问题转化为计算可行的精确近似解问题的一系列工具。
	数值方法在机器学习领域的应用包括但不限于:
	\begin{enumerate}
		\item 数值稳定性问题
		\item 迭代法求精确近似解
		\item 矩阵的特征分解及其应用
	\end{enumerate}
	本文主要将围绕上述三个主题总结和讨论数字方法在机器学习领域的应用。

	% 关键词用“英文逗号”分隔
	\thusetup{
		keywords = {数值方法, 机器学习},
	}
\end{abstract}

\chapter{概述}
\label{chap:intro}
数值方法是数学的一个分支,它涉及解决科学应用中出现的概率问题的数值算法的理论基础。
这门学科涉及到各种问题,从函数和积分的近似到代数方程、超越方程、微分方程和积分方程的近似解,特别强调数值算法的稳定性、准确性、效率和可靠性。
数值方法是一门贴近实际生活问题的一门课程,它试图以最有效,最现实的方式解决各种数学和工程问题。
从维基百科的介绍中\footnote{\url{https://en.wikipedia.org/wiki/List_of_numerical_analysis_topics}},我们可以看到数值方法几乎无处不在。
本文重点介绍数值方法在机器学习方向中的应用,在第~\ref{chap:num_cond}章中我们主要介绍在机器学习算法中,数值稳定性带来的计算问题,包括数值溢出以及病态条件。
因为机器学习的算法中往往存在大量的无法直接计算出符号解的问题,在第~\ref{chap:iter}章中我们将介绍如何使用迭代法将计算不可行的问题转化为计算可行的精确近似解求解问题。
接着,我们在第~\ref{chap:matrix_decom}章详细介绍矩阵特征值分解及在机器学习领域中的应用。
最后,第~\ref{chap:conclusion}章总结了数值方法中这三类问题在机器学习领域中的应用。

\chapter{数值稳定性问题}
\label{chap:num_cond}
由于计算机只能用有限精度来表示浮点数,这会造成数值存储和计算过程中存在大量的舍入误差和截断误差。
甚至在数字计算机上计算一个数学函数,当函数涉及实数时,因为不能用有限的内存来精确表示,该计算就会变得很困难。



\chapter{迭代法求精确近似解}
\label{chap:iter}
通过迭代来不断更新对某个数学问题的近视解,而不是直接得出一个可以求出符号解的公式。

\chapter{矩阵特征分解}
\label{chap:matrix_decom}
在机器学习和统计学中,我们经常要处理结构性数据,这些结构性数据一般用行和列的表格,或者矩阵来表示。
机器学习中的很多问题都可以使用矩阵代数和向量微积分来解决。
在这章中,我们将讨论一些可以使用矩阵分解技术解决的问题。
同时,还将讨论哪些特定的分解技术已经被证明对一些机器学习问题比较有效。

\chapter{结论}
\label{chap:conclusion}