\documentclass[fleqn,explicit,twoside,openany]{tufte-book}
\hypersetup{colorlinks}
\usepackage{xeCJK}
\usepackage[T1]{fontenc}

\font\ttfsegoeuilsmall Colaborate-Thin at12pt
\font\ttfsegoeuil Colaborate-Thin at16pt
\font\ttfsegoeuilhuge Colaborate-Thin at30pt

\renewcommand{\contentsname}{Table of Contents}

\title[A Note for courses in THUSZ]{\ttfsegoeuilhuge A Note for courses in THUSZ}
\author{\ttfsegoeuil DCMMC}

\usepackage{pdfpages}
\usepackage{amsthm}
\usepackage{xcolor}
\usepackage{hyperref}
%\usepackage{wrapfig}
% http://mirrors.ctan.org/macros/latex209/contrib/picins/picins.sty
\usepackage{picins}
\usepackage{soul}
\usepackage{lipsum}
\usepackage{booktabs}
\usepackage{graphicx}
\usepackage{algorithm}
\usepackage{algorithmic}
\usepackage{answers}
\usepackage[absolute,overlay]{textpos}
\usepackage{verbatim}
\usepackage{fancyvrb}
\usepackage{xspace}
% provides boldsymbol
\usepackage{bm} 
% provides inline frac: \nicefrac
\usepackage{nicefrac}
%\usepackage[labelsep=none]{caption}
\usepackage{tikz}
\usepackage{units}
\usepackage{enumitem} 
\usepackage{makeidx}
\usepackage{tabularx}
\usepackage{colortbl}
\usepackage{multirow}
\usepackage{calc}

\usepackage{haldefs}

\setkeys{Gin}{width=\linewidth,totalheight=\textheight,keepaspectratio}
\graphicspath{{graphics/}}
\fvset{fontsize=\normalsize}
\usetikzlibrary{shapes,snakes}

\include{halbook}

%https://github.com/Tufte-LaTeX/tufte-latex/issues/107#issuecomment-183679016
%Next block avoids bug, from  http://tex.stackexchange.com/a/200725/1913 
\ifx\ifxetex\ifluatex\else % if lua- or xelatex http://tex.stackexchange.com/a/140164/1913
\newcommand{\textls}[2][5]{%
	\begingroup\addfontfeatures{LetterSpace=#1}#2\endgroup
}
\renewcommand{\allcapsspacing}[1]{\textls[15]{#1}}
\renewcommand{\smallcapsspacing}[1]{\textls[10]{#1}}
\renewcommand{\allcaps}[1]{\textls[15]{\MakeTextUppercase{#1}}}
\renewcommand{\smallcaps}[1]{\smallcapsspacing{\scshape\MakeTextLowercase{#1}}}
\renewcommand{\textsc}[1]{\smallcapsspacing{\textsmallcaps{#1}}}
\fi

%\renewenvironment{mycomment}{}{}   % comment this if you want comments :)

\let\cleardoublepage\clearpage

\makeindex

% emph 中文强调命令
% 如果需要使用原义强调命令,则可使用 \emph* 命令
% Ref: https://wenda.latexstudio.net/article-5065.html
\usepackage{expl3}
\usepackage{etoolbox}
% 声中文明强调字体及强调方式
% 用\bfseries表示强调
\makeatletter
\newcommand*\emphfont{\normalfont\bfseries}
\DeclareTextFontCommand\@textemph{\emphfont}
\newcommand\textem[1]{%
	\ifdefstrequal{\f@series}{\bfdefault}
	{\@textemph{\CJKunderline{#1}}}% 用下划线表示强调中的强调
	{\@textemph{#1}}%
}
\makeatother


\ExplSyntaxOn
% % 修复 soul 的 hl 以修复 concept
% Ref: https://tex.stackexchange.com/questions/449236/soul-package-command-hl-can-not-work-with-escaped-space
\let\hlORIG\hl
\tl_new:N \l_jdhao_hlx_tl
\RenewDocumentCommand \hl { m }
{
	\tl_set:Nn \l_jdhao_hlx_tl { #1 }
	\tl_replace_all:Nnn \l_jdhao_hlx_tl { \  } { ~ }
	\regex_replace_all:nnN { \c[^CBEMTPUDA]\S } { \c{hbox} \0 } \l_jdhao_hlx_tl
	\exp_args:NV \hlORIG \l_jdhao_hlx_tl
}
% 重新定义 \emph 强调命令
% 参考:https://tex.stackexchange.com/questions/13048/upright-parentheses-in-italic-text?rq=1
\cs_new_eq:Nc \emph_old:n { emph~ } % 备份`\emph`命令
\cs_new_protected:Npn \emph_braces:n #1 % 排版括号为直立符号
{ \mode_if_math:TF {#1} { \textup{#1} } }

\cs_new:Npn \emph_new:n #1 {
	\tl_set:Nn \l_emph_tl {\textem{#1}}% 中文强调时用黑体
	\tl_replace_all:Nnn \l_emph_tl {(}{\emph_braces:n{(}}% 替换(为直立
	\tl_replace_all:Nnn \l_emph_tl {)}{\emph_braces:n{)}}% 替换)为直立
	\tl_replace_all:Nnn \l_emph_tl {[}{\emph_braces:n{[}}% 替换[为直立
	\tl_replace_all:Nnn \l_emph_tl {]}{\emph_braces:n{]}}% 替换]为直立
	\exp_args:NV \emph_old:n \l_emph_tl
}
\RenewDocumentCommand {\emph} {sm} {% \emph*是原命令,\emph为重新定义的命令
	\IfBooleanTF {#1} {\emph_old:n {#2}} {\emph_new:n {#2}}
}
\ExplSyntaxOff


\begin{document}
\setcounter{secnumdepth}{3}

\frontmatter

\maketitle
\tableofcontents
\listoffigures
\listoftables
\cleardoublepage
\mainmatter

% (DCMMC): 导入笔记
%\chapter{NLP with DL}\label{sec:nlp}

\chapternote{Natural Language Processing with Deep Learning}{Stanford CS224n Winter 2019}

\begin{learningobjectives}
	\item Word Vector
	\item Calculus Review
	\item RNN \& Language Model
	\item Seq2Seq \& Attention
	\item ConvNet for NLP
	\item Transformer
\end{learningobjectives}

\dependencies{Machine Learning Basic}


\section{Word Vector}

Arguably the most simple word vector, i.e., \concept{one-hot vector}: an $\R^{|V| * 1}$ vector with one $1$ and the rest $0$s.
Note that these one-hot vectors are \concept{orthogonal} (i.e., no similarity/relastionship) and $V$ is a very big vocabulary ($\sim 500k$ words for english).

\sidenotedcmmc{In traditional NLP (before 2013), words are regarded as discrete symbols (\concept{localist} representation) and cannot capture similarity. One-hot vector is an example.}

Another idea: \concept{distributional representation} in modern statistical NLP. A word's meaning is given by the words that frequently appear close-by.
Using some $N$-dim ($N \ll |V|$) space is sufficient to encode all semantics of our language into a dense vector.
Once we get the word embedding matrix where each column is a word vector, we can query the word vector from one-hot representation by treating it as \concept{lookup table} instead of using matrix product.

\Figure{word_analogy}{An example of word analogy of man:woman :: king:?}

To evaluate word vectors, there are two fold: \emph{intrinsic} (directly used, e.g. word analogies/similarity) and \emph{extrinsic} (indirectly used in real task, e.g. Q\&A).
Word vector analygies for  $a:b :: c:\textcolor{red}{d}$ is calculated by cosine similarity as example shown in Fig.~\ref{fig:word_analogy}:

\begin{equation}
d = \arg \max_i \frac{(x_b - x_a + x_c)^\top x_i}{\lVert x_b - x_a + x_c \rVert}
\end{equation}

If we have hundreds of millions of words, it's okay to start the vectors \emph{randomly}.
If there is a \emph{small} ($\le 100,000$) training data set, it's best to just treat the pre-trained word vectors as \emph{fixed}.
In the other hand, if there is a large dataset, then we can gain by \concept{fine tuning} of the word vectors.

\subsection{Word2vec}

Two families of models: \concept{Skip-gram} and \concept{Continuous Bag of Words}.

Idea of \concept{Skip-gram} (predicting context words by a given center word) in  Word2vec\mycite{word2vec}:

\begin{itemize}
	\item a large corpus of text $T$ with a vocabulary $V$
	\item every word is represented by a vector $w \in \mathbb{R}^d$ and start off as a random vector
	\item use the (cosine) similarity of the word vectors for $c$ (center word) and $o$ (context/outside word) to calculate the probability of $o$ given $c$: $p(w_o | w_c)$
	\item adjusting the word vectors to maximize the probability
\end{itemize}

\sidenotedcmmc{Why we use two vectors per word? Make it simpler to calculate the gradient of loss function. Because the center word would be one of the choices for the context word and thus squared terms are imported. Average both vectors at the end is the final word vector.}

The conditional probability is calculated by the \concept{softmax} (normalize to probability distribution) of \concept{cosine} similarity (review dot product: $\bm{a} \cdot \bm{b} = |\bm{a}||\bm{b}| \cos\left<\bm{a}, \bm{b}\right>$).
Note that the visualization of word vectos utilizes 2D projection (e.g. PCA) that will loss huge information.

\begin{equation}
p(w_o | w_c) = \frac{\exp(u_o^\top v_c)}{\sum_{w \in V}(u_w^\top v_c)}
\end{equation}
where $v_c$ denotes the center word vector of $w$ when $w$ is used as a center word in the formula, and $u_w$ denotes the context word vector of $w$ as the similar way.
A demo of the window size and conditional probability is shown in Fig.~\ref{fig:word2vec:window_cond}.

\MoveNextFigure{+5cm}
\Figure{word2vec:window_cond}{A demo of the window size and $p(w_o | w_c)$}


The objective function (a.k.a loss or cost function) is given by the (average) negative log likelihood (abbr. \concept{NLL}).
The parameters of the model are adjusted by minimizing the loss function $J(\theta)$ or maximizing the likelihood.
This is, give a high probability estimate to those words that occur in the context and low probability to those don't typically occur in the context.


\begin{align}
	\arg\max_\theta L(\theta) &= \prod_{c=1}^{T} p(w_{c-m}, \cdots, w_{c-1}, w_{c+1}, \cdots, w_{c+m} | w_c; \theta) \nonumber \\
	&= \prod_{c=1}^{T} \prod_{\substack{-m \le j \le m \\ o = j + c \\ o \ne c}} p(w_o | w_c; \theta) \nonumber \\
	&\Downarrow \nonumber \\
	\arg\min_\theta -\frac{1}{T} \log L(\theta) &= -\frac{1}{T} \sum_{c=1}^{T} \sum_{ \substack{-m \le j \le m \\ o=j+c \\ o \ne c}} \log p(w_o | w_c; \theta) \nonumber \\
	&= -\frac{1}{T} \sum_{c=1}^{T} \sum_{ \substack{-m \le j \le m \\ o=j+c \\ o \ne c}} \left( u_o^\top v_c - \log \sum_{w \in V} \exp(u_w^\top v_c)\right) 
	\label{eq:word2vec_loss}
\end{align}
where $m$ is the window size, $\theta \in \mathbb{R}^{2d|V|}$ represents all model parameters.
And we assume that $p(\cdot | w_c)$ are \concept{i.i.d}.

\sidenotedcmmc{The properties of $\log$ and $\arg \max$ ($\arg \min$) used in Eq.~\ref{eq:word2vec_loss} are VERY useful. $\exp(\cdot)$ ensures anything positive.}

We use \concept{gradient descent} (i.e. averaged gradient of all samples/windows) to optimize the loss function.
Note that stochastic (one sample/window with noisy estimates of the gradients) or mini-batch (a subset of samples/windows with size powered of $2$ such as $64$) gradient descent methods are useful to prevent overfitting and train for large dataset.
Calculating the gradient of the loss function is trivial:

\begin{equation}
\begin{split}
\frac{\partial J}{\partial v_c} &= -\frac{1}{T} \sum_{c=1}^{T} \sum_{ \substack{-m \le j \le m \\ o=j+c \\ o \ne c}} \left(u_o - \sum_{x \in V} \frac{\exp(u_x^\top v_c) u_x}{\sum_w \exp(u_w^\top v_c)}\right) \\
&= -\frac{1}{T} \sum_{c=1}^{T} \sum_{ \substack{-m \le j \le m \\ o=j+c \\ o \ne c}} \left(u_o - \sum_{x \in V} p(w_x | w_c) \cdot u_x \right)
\end{split}
\end{equation}

\begin{equation}
\frac{\partial J}{\partial u_o} = -\frac{1}{T} \sum_{c=1}^{T} \sum_{ \substack{-m \le j \le m \\ o=j+c \\ o \ne c}} \left(v_c - p(w_o | w_c)\right)
\end{equation}

Iteratively update equation (na\"ive version) is given by:

\begin{equation}
\theta^{new} = \theta^{old} - \alpha \nabla_\theta J(\theta)
\end{equation}
where $\alpha$ is the learning size (step size).

Note that the summation over $|V|$ ($\sum_{x \in V}$) is very expensive to compute!
For every training step, instead of looping over the entire vocabulary, we can just sample several negative examples!
\concept{negative sampling}: train binary logistic regression instead.
$p(D=1|w_o,w_c)$ denotes the probability when $(w_o,w_c)$ came from the same window pf the corpus data, and $p(D=0|w_o,\tilde{w}_o)$ is the probability given $(w_o,\tilde{w}_o)$ did not come from the same window (i.e. noisy/invalid pair).
Randomly sample a bunch of noise words from the \concept{unigram distribution} raised to the power of $3/4$: $p(w) = \nicefrac{U(w)^{3/4}}{Z}$, where $U(w)$ is the counts for every unique words (i.e. unigram) and $Z$ is the nomalization term.

To avoid high frequence effect of words such as \concept{of} and \concept{the}, one simple way is just lop off the first biggest component in the word vector.
The unigram with power of $3/4$ in word2vec is also a trick to handle the effect, where it decrease how often you sample very common words and increase how often you sample rare words.

The objective function is also come from NLL: 

\begin{equation}
J(\theta) = - \frac{1}{T} \sum_{c=1}^T \sum_{ \substack{-m \le j \le m \\ o=j+c \\ o \ne c}} \left( \log \sigma \left( u_o^\top v_c \right) + \sum_{j \sim p(w)} \left[ \log \sigma \left(-u_j^\top v_c\right) \right] \right)
\end{equation}
where \concept{sigmoid} function is $\sigma(x) = \frac{1}{1 + e^{-x}}$ which can be seen as the 1D (binary) version of softmax and used to output the probability, and $k$ is the number of negative samples such as $5$ and $15$.
Note that according to the symmetric property of sigmoid function we get: $P(D=0|\tilde{w}_j,w_c) = 1 - P(D=1|\tilde{w}_j,w_c) = \sigma \left(-u_j^\top v_c\right)$.

\sidenotedcmmc{Although word2vec model is fairly simple and clean, there are actually many tricks which aren't particularly theoretical.}

\concept{Continuous Bag of Words} (CBOW): predict center word from (bag of) context words.
Similar to Skip-gram, the objective function is formulated as:

\begin{align}
J &= -\frac{1}{T} \sum_{c=1}^{T} \log P (w_c | w_{c-m}, \cdots, w_{c-1}, w_{c+1}, \cdots, w_{c+m}) \\
&= -\frac{1}{T} \sum_{c=1}^{T} \log p (v_c | \hat{u}) \\
&= -\frac{1}{T} \sum_{c=1}^{T} \log \mathop{\textup{softmax}}\limits_{c}(v_c^\top \hat{u}) \\
&= -\frac{1}{T} \sum_{c=1}^{T} (v_c^\top \hat{u} - \log \sum_{j=1}^{|V|} \exp (v_j^\top \hat{u}))
\end{align}
where $\hat{u} = \frac{1}{2m} \sum_{ \substack{-m \le j \le m \\ o=j+c \\ o \ne c}} u_o$


Although word2vec can capture complex patterns beyond word similarity, it has inefficient usage of statistics (i.e. rely on sampling rather than directly use counts of words).

\subsection{HW1}

A simple intro to co-occurrence matrix, SVD, cosine similarity, and some applications (e.g. word analogy) of word2vec.

\subsection{GloVe}
\Figure{cooccurrence_matrix}{An example of co-occurrence matrix with window size of $1$}

Co-occurrence matrix $X \in \mathbb{R}^{|V| * |V|}$ with window size $k$.
Fig.~\ref{fig:cooccurrence_matrix} shows an example.
Note that such matrix is extremely sparse and very high dimensional, and the dimensions of the matrix change very often as new words are added very frequently and corpus changes in size.
We can perform SVD on $X$ to reduce the dimensionality to $25 \sim 1000$-dim.
In addition, there are some hacks to $X$ that transform the raw count introduced by \mycite{rohde2005_hacks}: (1) set upper bound (e.g. $100$) or just ignore them all for the counts of too frequent words, (2) ramped windows that count closer words more. (3) use Pearson correlations instead of counts.
Note that they made some interesting observation in their word vector that the verb (e.g. swim) and the corresponding doer (e.g. swimmer) pairs are roughly \emph{linear components} (e.g. $\bm{v}_{swimmer} - \bm{v}_{swim} = k (\bm{v}_{driver} - \bm{v}_{drive})$).

\tododcmmc{SVD}

Although the aforementioned conventional method has disproportionate importance given to large counts and mainly only capture word similarity, it enjoys the fast training and efficient usage of statistics.
GloVe (\textbf{Glo}bal \textbf{Ve}ctor) \mycite{GloVe} combines the advantages from both of this conventional method (global count matrix factorization) and the DL-based methods (local context window methods) such as word2vec.
It captures global corpus statistics directly.

\Figure{ratio_cooccurrence}{An example of the conditional probabilities and their ratio in GloVe paper.}

Some notations: $X_{ij}$ tabulate the number of times word $j$ occurs in the context of word $i$, $X_i = \sum_{k} X_ik$ is the number of times any word appears in the context of word $i$ i.e., the nomalization denominator.
$P_{ij} = P(j|i) = \nicefrac{X_{ij}}{X_i}$ is the probability that word $j$ appear in the context of word $i$.
The crucial insight is that the \emph{ratios} of co-occurrence probabilities as shown in Fig.~\ref{fig:ratio_cooccurrence} to encode meaning components.
We'd like to leverage the word vectors $w_i, w_j, \tilde{w}_k$ to represent such ratio: $F(w_i, w_j, \tilde{w}_k) = \nicefrac{P_{ik}}{P{jk}}$, where $\tilde{w}$ is a seperate \emph{context} word vector for various \emph{probe words} $k$, instead of the word vector $w$ (similar to center word vector in skip-gram).

We can select a unique choice of $F$ by enforcing a few desiderata (i.e. restrictions).
To fit the demand of the \emph{linear components} and the output \emph{scalar} value, in addition to the \emph{homomorphism}
between the groups $(\mathbb{R}, -)$ and $(\mathbb{R}^+, \div)$ (i.e., $F(i,j) = \nicefrac{P_{ik}}{P{jk}} = \nicefrac{1}{F(j,i)} = \nicefrac{P_{jk}}{P{ik}}$), we can derivate that $F(w_i, w_j, \tilde{w}_k) = F\left((w_i - w_j)^\top \tilde{w}_k\right) = \nicefrac{F(w_i^\top \tilde{w}_k)}{F(w_j^\top \tilde{w}_k)} = \nicefrac{P_{ik}}{P_{jk}}$.
Therefore, $F=\exp, w_i^\top \tilde{w}_k = \log(P_{ik}) = \log (X_{ik}) - \log (X_i)$.
Note that the symmetry property of co-occurrence: $X_{ik} = X_{ki}$.
We add two biases to restore the symmetry: $w_i^\top \tilde{w}_k + b_i + \tilde{b}_k = \log (X_{ik})$, where we can analogy that $b_i + \tilde{b}_j = \log X_{i}$.

More details, the relationship to the "global skip-gram" and the complexity refer to the original GloVe paper~\mycite{GloVe}.

\sidenotedcmmc{To handle the ill-defined $\log$ function when its argument be $0$ (its common that $X_{ij}=0$), the authors use the factorized log: $\log(X_{ik}) \rightarrow \log (1+X_{ik})$.}

\begin{align}
w_i \cdot w_j &= \log P(i|j) \\
w_x \cdot (w_a - w_b) &= \log \frac{P(x|a)}{P(x|b)}
\end{align}

Therefore, the ratios of co-occurrence probabilities is the \concept{log-bilinear model with vector differences}.
The final objective function is \emph{weighted} \concept{least squares} (MSE) for this regression problem.

\begin{equation}
	J = \sum_{i,j=1}^V f(X_{ij})\left(w_i^\top \tilde{w}_j + b_i + \tilde{b}_j - 
	log X_{ij}\right)
\end{equation}
where weighted function (is also a hyperparamter) is:

\begin{equation}
f(x) =
\begin{cases}
	\left(\frac{x}{x_{max}}\right)^\alpha & \text{if } x < x_{max} \\
	1 & \text{otherwise}
\end{cases}
\end{equation}
where $x_{max} = 100, \alpha = \nicefrac{3}{4}$ (\emph{empirical} value). 

\subsection{Word sense ambiguity}
Because most words have lots of meanings.
One crude way \mycite{huang-etal-2012-improving} is to cluster word windows around words, retrain with each
word assigned to multiple different clusters $\textsf{bank}_1$, $\textsf{bank}_2$, etc.
Another method \mycite{TACL_word_senses} is weighted sum of different senses of a word reside in a linear superposition, e.g.:

\begin{equation}
v_{\text{pike}} = \alpha_1 v_{\text{pike}_1} + \alpha_2 v_{\text{pike}_2} + \alpha_3 v_{\text{pike}_3}
\end{equation}
where $\alpha_i = \frac{f_i}{\sum_{j=1}^3 f_j}$ for frequency $f$.

The result is counterintuitive very well, because of the idea from \emph{sparse} coding you can actually separate out the senses.

\section{Math Backgrounds}
For \concept{multi-class classification} problem, \concept{NLL} (negative likelihood loss) is the objective function of \concept{Maximum Likelihood Estimate} (abbr, MLE):

\begin{equation}
J(\bm{\theta}) = - \sum_i \log p(y = y^{true}_i | \bm{x}_i; \bm{\theta})
\end{equation}

\concept{cross entropy} (distance measure) between (discrete) distribution $p$ and $q$ is more convenient way:

\begin{equation}
H(p, q) = - \sum_{c=1}^C p(c) \log q(c)
\end{equation}

However, in the multi-class (with single label) setting, the p(c) is the \concept{ground truth distribution} which has the \emph{one-hot} style (\concept{empirical distribution}), i.e. $p = [0, \cdots, 0, 1, 0, \cdots, 0]$ where $1$ at the right class and $0$ everywhere else.
Therefore, the \concept{cross entropy} in the multi-class classification is \emph{equal} to the NLL.

A simple $k$-class model example is \concept{dense layer} with \emph{softmax}:

\begin{equation}
	p(y|\bm{x}; \bm{\theta}) = softmax(\bm{W}_2 f(\bm{W}_1 \bm{x} + \bm{b}))
\end{equation}
where $\bm{\theta} = [\bm{W}_1, \bm{b}, \bm{W}_2]^\top $ are the parameters, $\bm{x} \in \mathbb{R}^m, \bm{W}_1 \in \mathbb{R}^{n * m}, \bm{b} \in \mathbb{R}^n, \bm{W}_2 \in \mathbb{R}^{k * n}$, $f(\cdot)$ is a kind of simple activate (non-linear) function to provide non-linearity, such as $ReLU(x) = max(0, x)$.
The visualization of neural network refer to \sidenote{ ConvNetJS: \url{https://cs.stanford.edu/people/karpathy/convnetjs/demo/classify2d.html}}.

The \concept{Jacobian Matrix} (generalization of the gradient) of function $\bm{f}(\bm{x}): \mathbb{R}^n \rightarrow \mathbb{R}^m$ is a $m \times n$ matrix: $\left(\frac{\partial \bm{f}}{\partial \bm{x}}\right)_{ij} = \frac{\partial f_i}{x_j}$.

Supposed that we have a function $\bm{g}(\bm{f}(x)), \bm{f}: \mathbb{R} \rightarrow \mathbb{R}^2, \bm{g}: \mathbb{R}^2 \rightarrow \mathbb{R}^2$, we can compute the partial derivative of $\bm{g}$ w.r.t $x$ by \concept{chain rule}:

\begin{equation}
	\frac{\partial \bm{g}}{\partial x} = \begin{bmatrix}
	\frac{\partial g_1}{\partial f_1} \frac{\partial f_1}{x} + \frac{\partial g_1}{\partial f_2} \frac{\partial f_2}{x}\\
	\frac{\partial g_2}{\partial f_1} \frac{\partial f_1}{x} + \frac{\partial g_2}{\partial f_2} \frac{\partial f_2}{x}
	\end{bmatrix}
\end{equation}

\sidenotedcmmc{$\frac{d g_1}{d \bm{y}} = \frac{\partial g_1}{y_1} + \frac{\partial g_2}{y_2}$ is the relationship of the full differential and the partial differential.}

It is the same as multiplying the two Jacobians:

\begin{equation}
\frac{\partial \bm{g}}{\partial x} = \frac{\partial \bm{g}}{\partial \bm{f}} \frac{\partial \bm{f}}{\partial x} = \begin{bmatrix}
\frac{\partial g_1}{\partial f_1} & \frac{\partial g_1}{\partial f_2} \\
\frac{\partial g_2}{\partial f_1} & \frac{\partial g_2}{\partial f_2}
\end{bmatrix} \begin{bmatrix}
\frac{\partial f_1}{\partial x} \\
\frac{\partial f_2}{\partial x}
\end{bmatrix}
\end{equation}

There are some useful identities:

\begin{itemize}
	\item $\frac{\partial \bm{x}}{\partial \bm{x}} = \bm{I}$
	\item $\frac{\partial \bm{Wx}}{\partial \bm{x}} = \bm{W}, \frac{\partial \bm{u}^\top \bm{x}}{\partial \bm{x}} = \bm{u}^\top$
	\item $\frac{\partial \bm{x^\top W}}{\partial \bm{x}} = \bm{W^\top}$
	\item For elemenwise function $\bm{f}(\bm{x})$: $\frac{\partial \bm{f}}{\partial \bm{x}} = \texttt{diag}(\bm{f}^\prime(\bm{x}))$
	\item $\frac{\partial \bm{\theta}^\top (\bm{W} \cdot \bm{h})}{\partial \bm{W}} = \bm{\theta} \bm{h}^\top$ where $\bm{\theta} \in \mathbb{R}^{D_\theta * 1}, \bm{W} \in \mathbb{R}^{D_\theta * D_h}, \bm{h} \in \mathbb{R}^{D_h * 1}$
	\item For cross entropy loss: $J(\bm{h}) = - \bm{y}^\top \log (\hat{\bm{y}}) = - \bm{y}^\top \log \texttt{softmax}(\bm{h})$ ($\bm{y}$ is one-hot vector) is: $\frac{\partial J}{\partial \bm{h}} = (\hat{\bm{y}} - \bm{y})^\top$
\end{itemize}

We can use \concept{backward propagation} (reversed of the \emph{topological sort}) and \emph{re-use} intermediate nodes to reduce complexity in the \emph{computation graph}.

Other machine learning basic concepts are: \concept{regularization} (e.g. L2) to prevent \concept{overfitting}, vectorization to parallelization, (non-linear) \concept{activation function} (e.g. sigmoid, tanh, (leaky) ReLU), parameter initialization (e.g. Xavier), \concept{Optimizer} (e.g. RMSprop, Adam), learning rate.

\subsection{Dropout}

\subsection{Xavier}

\subsection{Adam}

\subsection{Practice: Named Entity Recognition}
To find and classify words as entities (e.g. location, or organization) in text.
One simple idea is that train softmax classifier to classify a center word by taking
\emph{concatenation} of word vectors surrounding it in a window (\emph{word window}) \mycite{NER_ICML}.
To perform NER of localtion, we need (unnormalized) score for each window, and make \emph{true window}’s (i.e. location in the center) score larger and other \emph{corrupt window}’s score lower.
The model is formulated as:

\begin{equation}
s = \bm{W}_2 f(\bm{W}_1 \bm{x} + \bm{b})
\end{equation}


The objective function (\emph{max-margin loss}) is:

\begin{equation}
J = max(0, s_c - (s - 1))
\end{equation}
where $s$ and $s_c$ is the score of true window and corrupt window.
It ensure each window with an NER location at its center should have a score $+1$ higher than any window without a location at its center.

\subsection{HW2}

Gradient calculation and implementation of word2vec.

\textbf{1. Written: Understanding word2vec}
\begin{align*}
	&(a) \ \hat{y}_o = P(O = o | C = c) \\
	&(b) \ \frac{\partial J}{\partial \bm{v}_c} = (\hat{\bm{y}} - \bm{y})^\top \bm{U}^\top \\
	&(c) \ \frac{\partial J}{\partial \bm{U}} = \bm{v}_c (\hat{\bm{y}} - \bm{y})^\top \\
	&(d) \ \sigma (\bm{x}) = \frac{1}{1 + \exp (- \bm{x})}, \frac{d \sigma (\bm{x})}{\bm{x}} = \texttt{diag} (\sigma (x_i) (1 - \sigma (x_i))) \\
	&(e) \ \frac{\partial J}{\partial \bm{v}_c} = \sum_k \sigma (\bm{u}_k^\top \bm{v}_c) \bm{u}_k^\top - (1 - \sigma (\bm{u}_o^\top \bm{v}_c)) \bm{u}_o^\top \\
	& \ \frac{\partial J}{\partial \bm{u}_o} = (\sigma (\bm{u}_o^\top \bm{v}_c) - 1) \bm{v}_c^\top \\
	& \ \frac{\partial J}{\partial \bm{u}_k} = \sigma (\bm{u}_k^\top \bm{v}_c) \bm{v}_c^\top \\
	&(f) \ (i) \frac{\partial J}{\partial \bm{U}} = \sum_o \bm{v}_c (\hat{\bm{y}}_o - \bm{y}_o)^\top \\
	&(ii) \frac{\partial J}{\partial \bm{v}_c} = \sum_o (\hat{\bm{y}}_o - \bm{y}_o)^\top \bm{U}^\top \\
	&(iii) \frac{\partial J}{\partial \bm{v}_w} = \bm{0}
\end{align*}

\sidenotedcmmc{Use shape convention to check the result.}

\textbf{2 Coding: Implementing word2vec}

Note that $\bm{U}, \bm{V}$ in the handout are the matrices whose $i$-th column is the $n$-dimensional embedded vector for word $w_i$.
However, in the codes of HW2, all the centerWordVectors and outsideVectors are as rows.

\section{Dependency Parser}
Two views of linguistic structure: (1) constituency (i.e., phrase structure grammar, or context-free grammar) (2) Dependency structure.
Dependence parse trees (single root with optional fake root, acyclic) use binary asymmetric relations which depicted as typed arrows going from \emph{head} to \emph{dependent}.
Note that the natural language is ambiguity.

Basic transition-based dependency parser \mycite{nivre-2003-efficient} with stack $\sigma = [\text{ROOT}]$, buffer $\beta = w_1, \cdots, w_n$, set of dependency arcs $A = \emptyset$, and a set of actions (\emph{transitions}) based on the above $3$-tuple:
\begin{align}
&\text{1. Shift: } \sigma , w_i | \beta, A \Rightarrow \sigma | w_i, \beta, A \nonumber \\
&\text{2. Left-Arc reduction: } \sigma | w_i | w_j, \beta, A \Rightarrow \sigma | w_j, \beta, A \cup \{r(w_j, w_i)\} \nonumber \\
&\text{3. Right-Arc reduction: } \sigma | w_i | w_j, \beta, A \Rightarrow \sigma | w_i, \beta, A \cup \{r(w_i, w_j)\} \nonumber
\end{align}
where $r(w_j, w_i)$ denotes $w_i$ is the dependency of $w_j$ (e.g. $\text{nsubj}(\text{ate} \rightarrow \text{I})$). The finish state is: $\sigma = [w], \beta = \emptyset$.
How to select (search) the best choice among the exponential size of different possible parse trees is the problem.
In 1960s, they use \emph{dynamic programming algorithms} ($\mathcal{O}(n^3)$).
In paper \mycite{nivre-2003-efficient}, the authors predict each action by a discriminative classifier (e.g. SVM classifier) which is more efficient but the accuracy is fractionally below the state-of-the-art.

\subsection{Neural Dependency Parsing}
Compared with traditional sparse feature-based discriminative dependency parsers, the work by \mycite{chen-manning-2014-fast} utilizes \concept{feedforward neural network model} with simple \concept{dense layers} and the softmax layer to predict each transition.
The input features with embedding dimension $d$ are:

\begin{enumerate}
	\item $x^{w} \in \mathbb{R}^{d * N_w}$: The top $3$ words on the stack and buffer $s_1, s_2, s_3, b_1, b_2, b_3$; the first and second leftmost / rightmost children of the top two words on the stack $lc_1(s_i), rc_1(s_i), lc_2(s_i), rc_2(s_i), i = 1, 2$; the
	leftmost of leftmost / rightmost of rightmost children of the top two words on the stack $lc_1(lc_1(s_i)), rc_1(rc_1(s_i)), i = 1, 2$; In total, $N_w = 18$.
	\item $x^{t} \in \mathbb{R}^{d * N_t}$: The corresponding POS (Part-of-speech, e.g. noun, verb, adjective) tags for $S_{word}$, $N_t = 18$.
	\item $x^{l} \in \mathbb{R}^{d * N_l}$:  The corresponding arc labels of words, excluding those $6$ words on the stack/buffer, $N_l = 12$.
\end{enumerate}

\sidenotedcmmc{Note that we use a special \textbf{NULL} token for non-existent elements: when the stack and buffer are empty or dependents have not been assigned yet.}

The predicted class is the one of transitions (i.e. shift, left/right arc reduction): $p = \texttt{softmax}(\bm{W}_2 f(\bm{W}_1^w \bm{x}^w + \bm{W}_1^t \bm{x}^t + \bm{W}_1^l \bm{x}^l + \bm{b}_1))$, where $f(\cdot)$ is the activation function (e.g. ReLU, or $x^3$).
The number of class is $3$ when untyped reductions or $T * 2 + 1$ when typed reductions (e.g. left-arc reduction with type \emph{nsubj}).

\section{Language Modeling and Recurrent Neural Networks}

Language Modeling: given a sequence of words $\bm{x}^{(1)}, \cdots, \bm{x}^{(t)}$, compute the probability distribution of the next word at $\bm{x}^{(t+1)}$:
\begin{equation}
P(\bm{x}^{(t+1)} | \bm{x}^{(1)}, \cdots, \bm{x}^{(t)})
\end{equation}

The joint probability of a text is:
\begin{equation}
P(\bm{x}^{(1)}, \cdots, \bm{x}^{(T)}) = \prod_{t=1}^T P(\bm{x}^{(t)} | \bm{x}^{(t-1)}, \cdots, \bm{x}^{(1)})
\end{equation}

$n$-gram is a chunk of $n$ consecutive words: unigram, bigram, trigram, 4-gram, ...
$n$-gram language model is based on a simplifying assumption: $\bm{x}^{(t+1)}$ depends only on the preceding $n-1$ words with i.i.d.:
\begin{align}
P(\bm{x}^{(t+1)} | \bm{x}^{(t)}, \cdots, \bm{x}^{(1)}) &= P(\bm{x}^{(t+1)} | \bm{x}^{(t)}, \cdots, \bm{x}^{(t-n+2)}) \\
&= \frac{P(\bm{x}^{(t+1)}, \bm{x}^{(t)}, \cdots, \bm{x}^{(t-n+2)})}{P(\bm{x}^{(t)}, \cdots, \bm{x}^{(t-n+2)})}
\end{align}
where the $n$-gram and (n-1)-gram probabilities are calculated by \emph{counting}.
There are some \emph{sparsity problems} with the above $n$-gram models such as the numerator or denominator is zero.
Some tricks such as \emph{smoothing} (add small $\delta$ to the count) and \emph{backoff} (e.g. $4$-gram backoff to $3$-gram) are proposed to solve them.

\begin{figure}[!thp]
	\centerline{\includegraphics[width=8.0cm]{figs/RNN.png}}
	\caption{Principle of RNN}
	\label{RNN}
\end{figure}

\sidenotedcmmc{Note that for $n$-gram, increasing $n$ makes sparsity problems worse. Typically $n \le 5$.}

To process \emph{variable} length \concept{sequential input} such as text, \concept{Recurrent Neural Network} (RNN) is introduced.
As the principle of RNN shown in Fig.~\ref{RNN}: \emph{repeat} (i.e. \concept{unfold} or unroll) the same RNN cell for each time-step but with different input and previous \concept{hidden state}.
A vanilla RNN for language modeling is:
\begin{align}
\bm{h}^{(t)} &= \sigma \left(\bm{W}_h \bm{h}^{(t-1)} + \bm{W}_x \bm{x}^{(t)} + \bm{b}_1\right) \\
\hat{\bm{y}} &= P(\bm{x}^{(t)} | \bm{x}^{(t-1)}, \cdots, \bm{x}^{(1)}) \nonumber \\
&= \texttt{softmax}(\bm{U}\bm{h}^{(t)} + \bm{b}_2) 
\end{align}
where $\sigma(\cdot)$ is the activation function, and $\bm{h}^{(0)}$ is the initial (random or zero) hidden state.
The gradient w.r.t. the weight matrix is the \emph{sum} of the gradients w.r.t each time it appears using \concept{back-propagation through time} (BPTT, just as same as normal back-prop).
And the \concept{evaluation metric} for language modeling is \emph{perlexity} which is equal to the exponential of the cross-entropy losses:
\begin{align}
\text{perplexity} &= \prod_{t=1}^T \left(\frac{1}{P_{LM} (\bm{x}^{(t+1)} | \bm{x}^{(t)}, \cdots, \bm{x}^{(1)})}\right)^{\nicefrac{1}{T}} \nonumber \\
&= \exp\left(\frac{1}{T} \sum_{t=1}^T -\log \hat{\bm{y}}^{(t)}\right)
\end{align}

There are some other applications of RNN: part-of-speech tagging, named entity recognition, sentence classification, text generator, encoder module, etc.
The final feature can be the final hidden state or elemen-wise max/mean of all hidden states.
However, the \emph{vanilla} RNN has these disadvantages: (1) recurrent computation is slow (2) hard to access long-term information (\concept{long-term dependencies}) due to \emph{gradient vanish} and \emph{gradient explosion}.
\chapter{工程硕士数学(a.k.a. 数值分析)}
\label{chap:numerical_analysis}

\chapternote{主要内容是讲解数值问题和求解方程}{工程硕士数学@Winter 2020}

\begin{learningobjectives}
	\item 误差,有效数值
	\item 数值稳定性
\end{learningobjectives}

\dependencies{先修课程:线性代数}

\section{绪论}

\subsection{计算机上的数}

采用\emph{浮点数}表示,尾数和阶数都具有限精度的位数,超过其位数会被\emph{截断}。

\concept{病态问题}:原始数据的微小变化会引起计算结果的巨大变化。

\subsection{误差与有效数字}

误差来源:

\begin{enumerate}
	\item 数学模型
	\item 观测误差
	\item 截断误差
	\item 舍入误差
\end{enumerate}

设 $x, x^*$ 分别为准确值和近似值,对 $\epsilon = |x - x^*| \le \sigma(x^*), \epsilon_r = |x - x^*| / |x^*| \le \sigma_r(x^*) = \sigma(x^*) / |x^*|$,$\epsilon$ 和 $\sigma(x^*)$ 分别为\concept{绝对误差}和\concept{绝对误差界},$\epsilon_r, \sigma_r$ 为\concept{相对误差}和\concept{相对误差界}。

将 $x^*$ 转化为类似于科学技术法的形式:
% (DCMMC): use align* to write multiline equations, and use `&` to state the
% alignment points.
% (DCMMC): use `\ ` to insert spaces in math equations.
\begin{align*}
	&|x - x^*| = \pm 10^k \times 0.a_1a_2 \cdots c_n \cdots \le 0.5 \times 10^{k-n} \\
	&s.t.\ a_1 \ne 0,  1 \le a_i \le 9, a_i \in \N, k \in \Z, n \in \N
\end{align*}
其中 $\Z$ 表示整数集,$n$ 表示\concept{有效数字位数}。

\concept{函数误差}:
\begin{align*}
	|f(x) - f(x_A)| \le |f^\prime (x_A)||x - x_A|
\end{align*}
其中 $x_A$ 是 $x$ 的近似值。

数值计算中常见误差的解决方法:
\begin{enumerate}
	\item 两个相近的数相减:使用分子有理化将减转化为加
	\item 避免大数吃小数
	\item 避免除数的绝对值远小于被除数的绝对值
	\item 简化运算次数
\end{enumerate}

\concept{数值稳定性}:

如果算法的初始值有误差,在运算中误差无限增加,不能控制,则该算法是数值不稳定的,反之则是数值稳定的。

\section{线性代数复习}

对角阵,三角阵的乘法和逆矩阵还是对角阵,三角阵。

正定阵:

矩阵的特征值:方程 $(\lambda I-A)x = \boldsymbol{0}$的非零解 $X\ne\boldsymbol{0}$ 就是 $A$ 的特征向量。

如果特征值均大于 $0$,则矩阵为正定阵。

Guass 消去法解方程的公式 2.3 不需要记.

\chapter{随机过程}
\label{chap:stochastic_processes}

\chapternote{主要内容是讲解随机过程}{随机过程@Winter 2020}

\begin{learningobjectives}
	\item 概率论复习
	\item 随机过程的定义
\end{learningobjectives}

\dependencies{先修课程:线性代数,概率论}

\section{概率论复习}
\concept{数学期望}:
\begin{align*}
	\mu &= EX = \int_{-\infty}^\infty x \mathrm{d} F(x) \\
	E[g(X)] &= \int g(x) \mathrm{d} F_X(x)
\end{align*}


实用公式:
\begin{enumerate}
	\item $E[aX] = aE[X]$,对于常数 $a$。
	\item $X$, $Y$ 相互独立 $\Rightarrow P(X, Y) = P(X) P(Y) \Rightarrow E[XY] = E[X]E[Y]$。
\end{enumerate}

\concept{方差(二阶矩)}:
\begin{align*}
	\sigma = DX := E[(X - EX)^2] = EX^2 - (EX)^2
\end{align*}

\concept{协方差}:
\begin{align*}
	C(X, Y) := E[(X - EX)(Y - EY)] = E[XY] - E[X]E[Y]
\end{align*}
用于多个 r.v. 之和的方差计算。

\concept{相关系数}:
\begin{align*}
	R(X,Y) := \rho(X,Y):=\frac{C(X,Y)}{\sigma_X \sigma_Y}
\end{align*}

\concept{复 r.v.}, 对实 r.v. $\eta, \zeta$:
\begin{align*}
	\xi &:= \eta + j \zeta, j := \sqrt{-1} \\
	E\xi &:= E\eta + j E\zeta, D\xi := E|\xi - E\xi| = E(\xi - E\xi)\overline{(\xi - E\xi)}
\end{align*}

\sidenotedcmmc{实用复数公式:$j^2 = -1, a \in \R, b \in \R, z = a +jb, |z| = \sqrt{a^2+b^2}, \overline{z} = a \textcolor{red}{-} jb, z\cdot \overline{z} = |z|^2, \exp(jx) = \cos(x) + j \sin(x), \frac{\partial jx}{\partial x} = j$}

\concept{Schwarz 不等式}(\emph{重点}):
\begin{align*}
	(E[XY])^2 \le E[X^2]E[Y^2]
\end{align*}

\emph*{Pf.}
\begin{align*}
	&E[(X-\alpha Y)^2] = E[X^2] - 2\alpha E[XY] + \alpha^2E[Y^2]>= 0 \\
	&\text{let } \alpha = \frac{E[XY]}{E[Y^2]} \\
	&\Rightarrow (E[XY])^2 \le E[X^2]E[Y^2]
\end{align*}

\sidenotedcmmc{母函数和特征函数这样定义就是为了方便计算,它们能够提供良好的性质:e.g. 母函数可以结合泰勒展开,简化计算均值和方差的计算。}

% TODO(DCMMC): 公式在 \concept 里面
\concept{母函数}$G(S)$(重点):
\begin{align*}
	&r.v.\ \xi, p_k := P{\xi = k}, k=0,1,2,\cdots\\
	&G_\xi(s) := Es^\xi = \sum_{k=0}^\infty p_k s^k
\end{align*}

性质:
\begin{enumerate}
	\item $p_k = G^{(k)}(0)/k!$,$k$阶导
	\item $E\xi = G^\prime(1), D\xi = G^{(2)}(1) + G^\prime(1) - (G^\prime(1))^2$
	\item $\xi_1, \cdots, \xi_n$ 相互独立,$\eta = \sum_{k=1}^n \xi_k, \Rightarrow G_\eta(s) = \prod_{k=1}^n G_{\xi_k}$
\end{enumerate}

常用 Taylor 公式:
\begin{align*}
	\exp(x) &= \sum_{k=0}^{+\infty} \frac{x^k}{k!}
\end{align*}

\sidenotedcmmc{母函数适用于离散型 r.v., 特征函数适用于连续型}

\concept{特征函数}$\Phi(t)$(核心就是 Fourier 变换):
\begin{align*}
	\Phi_\xi(t) := G_\xi(\exp(\mathit{j}t)) = \int_{-\infty}^\infty \exp(jt\xi) \mathrm{d}F(x)
\end{align*}

性质:
\begin{enumerate}
	\item 共轭对称:$\Phi(-t) = \overline{\Phi(t)}$
	\item 若 $\xi$ 的 $n$ 阶矩存在 $\Rightarrow \Phi_{\xi}^{(k)}(0) = j^k E\xi^k$
	\item $EX = j^{-1} \Phi^\prime(0), DX = - \phi^{(2)}(0) - (EX)^2$
	\item \concept{非负定性}(\emph{重点}):对$\forall \lambda_i \in \mathbb{C}, \forall t_1,\cdots,t_n \in \R, \Lambda = (\lambda_1, \cdots, \lambda_n)^\top \in \mathbb{C}^{n}, R_{ij} := \Phi_\xi(t_i - t_j)$,有 $\Lambda R \overline{\Lambda} = \textcolor{red}{\sum_i \sum_k \lambda_i \int \exp(j(t_i - t_j)\xi)\mathrm{d}F(x) \overline{\lambda_j}} = \textcolor{red}{\int \sum_i \lambda_i \Phi(t_i) \sum_j \overline{\Phi(t_j)\lambda_j} = \int \left(\sum_i \lambda_i \Phi(t_i)\right)^2}\ge 0$(i.e., $R\succeq 0$)。
\end{enumerate}

多维 r.v. 的特征函数:
\begin{equation}
	\Phi(t_1, \cdots, t_n) := E \exp(j(\textcolor{red}{t_1 \xi_1, \cdots, t_n \xi_n}))
\end{equation}

\section{随机过程(r.p.)}

Def:

设 $\{\Omega, \mathcal{F}, P\}$ 为概率空间,$T$ 为参数集,若对 $\forall t \in T$,$\xi(t)$ 是一个 r.v., 则称 r.v. 族 $\{\xi(t), t\in T\}$ 为该概率空间上的\concept{随机过程}。

固定样本点 $w_0$, $\xi_t(w_0)$ 为一个关于 $t$ 的确定性函数,称\concept{样本函数}。

\sidenotedcmmc{做题的时候我们可以固定 $t$ 来理解题目}

对 $\{\xi(t), t \in T\}, \forall t_1, \cdots, t_n \in T,$,其 $n$ 维分布函数:$F_{t_1,\cdots,t_n}(x_1, \cdots, x_n) := F(x_1, \cdots, x_n; t_1, \cdots, t_n) := P(\xi(t_1)\le x_1,\cdots, \xi(t_n)\le x_n)$。

分类:
\begin{enumerate}
	\item $T$ 离散集:随机序列/时间序列
	\item $T$ 连续集:随机过程
\end{enumerate}

对 r.p. $\xi(t)$ 定义一些数字特征(期望,二阶矩,自相关函数,自协方差函数,相关系数):
\begin{align*}
	\mu(t) &:= \int x \mathrm{d}F(x,t) \\
	\sigma(t) &:= E\xi^2(t) - (E\xi(t))^2 \\
	R(t_1,t_2) &:= E\xi(t_1)\xi(t_2) \\
	C(t_1,t_2) &:= E[(\xi(t_1) - \mu(t_1))(\xi(t_2) - \mu(t_2))] = R(t_1, t_2) - \mu(t_1)\mu(t_2)\\
	\rho(t_1,t_2) &:= \frac{C(t_1,t_2)}{\sigma(t_1)\sigma(t_2)}
\end{align*}

\sidenotedcmmc{r.p. 的均值和方差都是 $t$ 的确定性函数而不是一个值。}

例题:对二项过程 $\{Y(t) = X_1 + \cdots + X_t, t \in T, X_i \sim B(1, p)\}$,有:
% DCMMC: align* 环境中 `&&` 用于展示注释。
\begin{align*}
	\mu_Y(t) &= E[X_1+\cdots + X_t] = tp \\
	\sigma_Y(m,n) &= E[Y(m)Y(n)] - EY(m)EY(n) \\
	&= E\left[\textcolor{red}{\sum_{i=1}^{m} X_i^2 + \sum_{i\ne j}X_i X_j}\right] - mnp^2 &&(\text{w.l.o.g. Let } m = \min(m,n))\\
	&= mp - m(n-1)p^2 - mnp^2 \\
	&= \min(m,n)p(1-p)
\end{align*}
核心思想(\emph{重点})就是明确 $EX_i^2$ 和 $EX_iX_j$是不一样的,并且他们分别有 $m$ 和 $m(n-1)$ 个。

\concept{复随机过程}:

$\{X(t), t \in T\}, \{T(t), t \in T\}$ 两个实过程具有相同的参数集 $T$ 和概率空间,称 $Z(t) = X(t) + jY(t)$ 为复过程。

\begin{align*}
	\mu(t) &:= EZ(t) = EX(t) + jEY(t) \\
	C(t_1,t_2) &:= E|Z(t)|^2 - |EZ(t)|^2
\end{align*}

复 r.p. $Z(t) = X(t) +j Y(t)$, 其中 $X,Y$ 为实过程:
\begin{align*}
	\mu(t) &:= EX(t) + jEY(t) \\
	R(t_1,t_2) &:= EZ(t_1)\overline{Z(t_2)} \\
	DZ(t) &:= E|Z(t) - \mu(t)|^2 \\
	C(t_1,t_2) &:= E[(Z(t_1) - \mu(t_1))\overline{(Z(t_2) - \mu(t_2))}]
\end{align*}
同样的对复 r.p. 的自相关函数也有非负定性 $R_{ij}:= R(t_i,t_j), R \succeq 0$.

\subsection{二阶矩过程}

Def. 对 r.p. $\xi(t)$, $\forall t \in T, E|\xi(t)|^2 \le \infty$.

\concept{常用不等式}:
\begin{enumerate}
	\item $E|\xi + \eta| \le E|\xi| + E|\eta|$
	\item $E\xi \le |E\xi| \le E|\xi|$
	\item $E|\xi\eta| \le \sqrt{E|\xi|^2E|\eta|^2}$ (Schwartz 不等式)
	\item $\sqrt{E|\xi+\eta|^2} \le \sqrt{E|\xi|^2} + \sqrt{E|\eta|^2}$ (三角不等式)
\end{enumerate}

\emph*{Pf.} of 三角不等式:
\begin{align*}
	E|\xi+\eta|^2 &= E\xi\overline{\xi} + E\xi\overline{\eta} + E\overline{\xi}\eta + E\eta\overline{\eta} \\
	&\le E|\xi|^2 + 2E|\xi\overline{\eta}|+E|\eta|^2 && (\xi\overline{\xi} = |\xi|^2,  \text{ for } a = x +jy,\\
	& &&a + \overline{a} = 2x \le 2|a| = 2\sqrt{x^2+y^2}) \\
	&\le E|\xi|^2 + E|\eta|^2 + 2\sqrt{E|\xi|^2E|\eta|^2} && \text{(Schwartz 不等式)}\\
	&= \sqrt{E|\xi|^2} + \sqrt{E|\eta|^2}
\end{align*}

两个 r.p. 之间:
\begin{enumerate}
	\item 互相关函数 $R_{\xi\eta} := E\xi(t_1)\eta(t_2) = \int \int x y \mathrm{d} F_{\xi\eta}(x,t; t_1, t_2)$
	\item 互协方差 $C_{\xi\eta}(t_1,t_2) := E[(\xi(t_1) - \mu_\xi(t_1))(\eta(t_2) - \mu_\eta(t_2))]$
\end{enumerate}

复二阶矩 r.p.

二阶矩空间:$H:=\{\xi:E|\xi|^2\le\infty\}$ 即二阶矩存在的 r.v. 全体。

性质:
% (DCMMC): LaTeX debug 经验:经常因为打错字符报看不懂的错误,这时候可以从上次正常编译的版本到现在的增量代码入手,
% 可以采用二分回退(也就是删除掉一半新代码)搜索问题行。
\begin{enumerate}
	\item 线性空间,有三角不等式证明
	\item 定义范数(距离空间):$\lVert \xi\rVert := \sqrt{E|\xi|^2} := R_\xi(t,t), d(\xi,\eta) := \lVert \xi - \eta \rVert$, 范数算子满足:
	\begin{enumerate}
		\item $\lVert\xi\rVert \ge 0$
		\item $\lVert c\xi \rVert= |c|\lVert \xi \rVert$
		\item $\lVert \xi + \eta \rVert \le \lVert \xi \rVert + \lVert \eta \rVert$, 三角不等式
	\end{enumerate}
\item 定义内积(内积空间/Hilbert空间):$\langle\xi,\eta\rangle := E\xi\overline{\eta} = R(\xi, \eta)$,特性(大多可直接按定义证明):
	\begin{enumerate}
		\item $\langle\eta,\xi\rangle = \overline{\langle\xi,\eta\rangle}$, 注意顺序互换了
		\item $\langle c\xi, \eta\rangle = c\langle\xi,\eta\rangle, c \in \R$
		\item $\langle\xi,\xi\rangle = E|\xi|^2\ge0$ 且 $\langle \xi, \xi \rangle = 0 \Leftrightarrow \xi = 0, a.e.$ (a.e. 表示\emph{几乎处处})
		\item 柯西不等式(a.k.a. Schwartz 不等式):$|\langle \xi, \eta\rangle| \le E|\xi\overline{\eta}| \le \lVert\xi\rVert\lVert\eta\rVert$
	\end{enumerate}
\end{enumerate}

\sidenotedcmmc{为了满足相应概率上的性质,复数值的乘积一般写为 $a\overline{b}$ 而不是 $ab$。}

r.v. 的收敛性, 对$\{\xi(t) \in H\}, \exists \xi \in H$:
\begin{enumerate}
	\item 概率 $1$ 收敛(强收敛) $\xi_t \xrightarrow{a.s.} \xi: P(\lim_{t\to\infty} = \xi) = 1$
	\item \concept{均方收敛}(强收敛, \emph{重点}, 记作 $\text{l.i.m.}_{t\to\infty}\xi_t = \xi$) $\xi_t \xrightarrow{m.s} \xi: \lim_{t\to \infty} E|\xi_t - \xi|^2 = \textcolor{red}{\lim_{t\to\infty}\lVert\xi_t-\xi\rVert = 0}$
	\item 依概率收敛(弱收敛) $\xi_t \xrightarrow{P} \xi: \forall \epsilon > 0, \lim_{t\to \infty} P(|\xi_t - \xi| > \epsilon) = 0$
	\item 依分布收敛(弱收敛) $\xi_t \xrightarrow{P} \xi: \lim_{n\to\infty} F(x, t) = F(x)$
\end{enumerate}
注意 $\xi_ti, \xi$ 分别是 r.p. 和 r.v., 并且上面四个收敛的强弱程度从上往下递减(\emph{强收敛的 r.p. 当然也是弱收敛}), 不过因为概率 $1$ 收敛条件比较苛刻所以我们一般使用均方收敛.

\sidenotedcmmc{记住一定不要把 $\text{l.i.m.}$ 混淆成 $\lim$, 考试写错直接没分.}

均方收敛的重要性质 $\xi_n \xrightarrow{m.s.} \xi, \eta_n \xrightarrow{m.s.} \eta$:
\begin{enumerate}
	\item 均值收敛(\emph{积分极限可互换}) $\lim_{n\to\infty}E\xi_n = E\xi =  E[\text{l.i.m.}_{n\to\infty} \xi_n]$
	\item 范数收敛 $\lVert\xi_n\rVert \rightarrow \lVert\xi\rVert$
	\item 内积收敛 $\langle\xi_m,\eta_n\rangle \rightarrow \langle\xi,\eta\rangle$ (也就是 $\lim_{m,n\to\infty} E\xi_m\overline{\eta_n} = E\xi\overline{\eta}$)
	\item 线性 $a\xi_n+b\eta_n\xrightarrow{m.s.} a\xi+b\eta$
\end{enumerate}

\emph*{Pf.} of 均值收敛:
\begin{align*}
	|E\xi_n - E\xi| &= |E[\xi_n - \xi]| \\
	&\le E|\xi_n - \xi| \\
	&\le \sqrt{E|\xi_n - \xi|^2} &&\text{(Schwartz 不等式, 当}P\{Y=1\}=1\text{的特例)}\\
	&= \sqrt{\lVert \xi_n - \xi\rVert} \\
	\xi_n \xrightarrow{m.s.} \xi &\Rightarrow \lim_{n\to\infty}\lVert\xi_n-\xi\rVert  =0\\
	&\Rightarrow \lim_{n\to\infty} \sqrt{\lVert\xi_n-\xi\rVert} &&(\lim\sqrt{f(x)} = \sqrt{\lim f(x)}) = 0\\
	&\Rightarrow \lim_{n\to\infty}|E\xi_n-\xi| = 0
\end{align*}

\emph*{Pf.} of 内积收敛:
\begin{align*}
	E\xi_m\overline{\eta_n} - E\xi\overline{\eta} &= E[\xi_m\overline{\eta_n} - \xi\overline{\eta}] \\
	&= E[(\xi_m - \xi)(\overline{\eta_n - \eta}) + (\xi_m - \xi)\overline{\eta} + \xi(\overline{\eta_n - \eta})] \\
	&\le \sqrt{\textcolor{red}{E|\xi_m-\xi|^2E|\eta_n-\eta|^2}} + \sqrt{E|\xi|^2\textcolor{red}{E|\eta_n - \eta|^2}} + \sqrt{\textcolor{red}{E|\xi_m-\xi|^2}E|\eta|^2} \\
	&\xrightarrow[m\to\infty]{n\to\infty} 0
\end{align*}
还有一种思路就是 $\xi_m(\overline{\eta_n - \eta}) + (\xi_m - \xi)\overline{\eta}$.

\emph*{Pf.} of 范数收敛:
\begin{align*}
	E|\xi_n|^2 - E|\xi|^2 &= E[(|\xi_n| + |\xi|)(|\xi_n| - |\xi|)] \\
	&\le \sqrt{E(|\xi_n| + |\xi|)^2}\sqrt{E(|\xi_n| - |\xi|)^2} \\
	&\le \sqrt{E|\xi_n|^2E|\xi|^2}\sqrt{E(|\xi_n| - |\xi|)^2}
\end{align*}

\sidenotedcmmc{按照师兄经验, 考试基本考作业原题, 不考推导, 切勿落入推导公司的深坑!}

证明上述这几个定理的核心思想就是凑 $\lim\lVert\xi_n-\xi\rVert$.

% (DCMMC): LaTeX 真是 typo 杀手... 我把 exists 打错成 exsists 编译报错找了半天才发现...
\emph*{Pf.} of $\lim_{x\to a}\sqrt{f(x)} = \sqrt{\lim_{x\to a} f(x)}$ (是我太菜了...):
\begin{align*}
	\lim_{x\to a} f(x) &= L \Leftrightarrow \\
	\forall \epsilon_0 > 0, &\exists \sigma_0 > 0, \text{ s.t. } |x-a| \in (0, \sigma_0): \\
	&|f(x) - L| < \sigma_0 \\
	&\Downarrow \\
	\forall \epsilon_1 \textcolor{red}{= \frac{\epsilon_0}{\sqrt{L}}}, &\exists \sigma_1 \textcolor{red}{= \sigma_0}, \text{ s.t. } |x-a|\in(0,\sigma_1): \\
	|\sqrt{f(x)} - \sqrt{L}| &= \frac{|f(x) - L|}{\sqrt{f(x)} + \sqrt{L}} \\
	&\le \frac{|f(x) - L|}{\sqrt{L}} \\
	&< \frac{\epsilon_0}{\sqrt{L}} = \epsilon_1
\end{align*}


证明均方收敛的两个准则(\emph{重点}):

\concept{柯西准则}: $\{\xi_n\} \text{ is Cauchy列} \Leftrightarrow \lim_{m,n\to\infty}\lVert\xi_m - \xi_n\rVert^2 = 0 \Leftrightarrow \exists \xi \in H, \xi_n \xrightarrow{m.s.} \xi$.

\concept{Loève 准则}: $\xi_n \xrightarrow{m.s.} \xi \Leftrightarrow \lim_{m,n\to\infty}\langle\xi_m,\xi_n\rangle = c$, 其中 $c$ 是一个\emph{常数}(不是一定 $0$!)


对 $t_1 < \cdots t_n, t_i \in T, i \in [1, n]$,若增量
\begin{equation}
	X(t_1), X(t_2) - X(t_1), \cdots, X(t_n) - X(t_{n-1})
\end{equation}
相对独立,则 $\{X(t), t\in T\}$ 为\concept{独立增量过程}。若对一切 $0 \le s < t$,增量 $X(t) - X(s)$ 的分布只依赖 $t-s$, 则 $X_T$ 有\concept{平稳增量}。


宽平稳过程:
\begin{enumerate}
	\item $R(t_1, t_2) = R(0, t_2 - t_1) := R(t_1 - t_1)$
\end{enumerate}

平稳过程默认指宽平稳。

证明宽平稳:求均值和相关函数, 然后二阶矩就是 $R(0)$, 三个条件都要写全。

\chapter{语音信号数字处理}
\label{chap:speech_processing}

\chapternote{语音信号数字处理}{Winter 2020}

\begin{learningobjectives}
	\item 绪论
	\item 语音学基础
\end{learningobjectives}

\dependencies{无}

\section{绪论}

声波的物理描述:
\begin{enumerate}
	\item 音高 Pitch/音调: 由频率决定,单位:赫兹Hz。
	\item 音色 Timbre: 由声源本身的材料结构确定,说到底就是各个频率分量的振幅(能量)不同。
	\item 音强 Loudness/响度: 由声音增幅及离声源的距离决定,单位:分贝dB。
\end{enumerate}

\section{语音学基础}

语音产生:
\begin{enumerate}
	\item 声道
	\item 发音器官:肺,声带,软腭,阴腭,舌,牙齿,唇
	
\end{enumerate}

声带(声门)$\Rightarrow$音高$\Rightarrow$基频(源)。

\sidenotedcmmc{人的有些发音的时候声带是不震动的(清音,voiceless)。}

声道(鼻腔,口腔,鼻腔,舌头,etc)$\Rightarrow$共振特性$\Rightarrow$音色/内容(调制,\concept{滤波器})。

肺部压缩空气力量的大小$\Rightarrow$响度。

\begin{figure}[!h]
	\centerline{\includegraphics[width=8.0cm]{figs/speech_source_filter_model.png}}
	\caption{Source Filter Model.}
	\label{fig:Source_Filter_Model}
\end{figure}

\concept{Source-Filter Model}:

Source Spectrum(谐波)$\Rightarrow$Filter Function(每个峰值代表一个共振频率,提高/降低不同频率的增幅,不会改变频率成分) $\Rightarrow$Output Energy Spectrum.

\sidenotedcmmc{谐波就是不同频率的波叠加在一起,随着各个谐波分量频率的增加,其分量的分贝值也在下降。}

浊音 Voiced Speech:

由声带震动引起,语音波形(声门波,EGG Signal)具有明显的周期性。声带振动的频率称为基频($F_0$),人们可感受到稳定的音高存在。\emph{基频就是谐波分量中频率最低/周期最长的分量的频率}。

清音 Voiceless/Unvoiced Speech:

声带不振动,波形类似白噪声,人们无法感受到稳定的音高存在。

All languages use pitch to express emotionaland other paralinguisticinformation(超语言学信息), and to convey emphasis, contrast, and other such features in what is called intonation(语调)。

Phoneme 音位/音素:发音时不可分割的、最小的音位学单位,国际音标中每一个音标就是一个音位。

Morpheme 语素:最小的、具有语义的结构单元,是最小的语法单位,是最小的语音语义结合体。例如英语的 un-break-able,或者中文中的词语。这是 NLP 的东西,不是语音学的重点。

Viseme 视位/视素:
A visemeis a representational unit used to classify speech sounds in the visual domain, corresponding to the phoneme in the aural domain.
其实就是嘴型这种。

Bimodal Processing 双模态处理:

声音最好同时结合听觉和视觉,因为有时候同一个发音,在只听声音,只看唇形和两者结合随着三种情况下在人的感知里是三种声音(MkGurkEffect)。

Spectrum 语谱:

The spectrum of a signal is a representation of each of its frequency components and their amplitudes(振幅).

Spectrogram 语谱图:

A spectrogram is a way of envisioning how the different frequencies that make up a waveform change over time.

Wide-band Spectrogram 宽带语谱图:

频率分辨率取300-400Hz,时间分辨率2-5ms,良好的时间分辨率,频率分辨率较差。

Narrow-band Spectrogram 窄带语谱图:

频率分辨率取50-100Hz,时间分辨率5-10ms,良好的频率分辨率,时间分辨率较差。

共振峰:

是指在声音的频谱中能量相对集中的一些区域(语谱峰值)。
声音在经过共振腔时,受到腔体的滤波作用,使得频域中不同频率的能量重新分配。

\emph{第一和第二共振峰($F_1$和$F_2$) 对于区分不同元音尤为重要。}
\chapter{大数据分析}
\label{chap:big_data_analysis}

\chapternote{两次平时小作业:$25\%\times2$, 一次课程大作业(机器翻译或推荐系统): $50\%$}{吴志勇老师@Winter 2020}

\begin{learningobjectives}
	\item 绪论
	\item 数据统计分析数学基础
	\item 分析与处理方法
	\item 分布式与并行计算
	\item 前沿
\end{learningobjectives}

\dependencies{无}

\section{数据抽样和假设检验}

例:产品品控的检验:随机抽样来求次品率。

统计推断:Statistical inference is the act of generalizing from a sample (抽样) to a population (总体) with calculated degree of certainty (置信度).
e.g. $\overline{x} \rightarrow \mu$.

Metric: Precision and Reliability.

Sampling Distribution of a Mean ($\overline{x}$) (SDM).

Central Limit Theorem, unbiasedness (无偏估计), seqaure root law (估计方差: $\sigma_{\overline{x}} = \frac{\sigma_{x_i}}{\sqrt(n)}$).

假设检验:

小概率推断: $1 < \alpha \le 0.05$

$\alpha$ 显著性水平, $Z$ 统计量, 统计假设 $H_0,H_1$.


\bibliographystyle{apalike}
\bibliography{bibfile}

{
\linespread{1}
\printindex
}

\end{document}
