\chapter{随机过程}
\label{chap:stochastic_processes}

\chapternote{主要内容是讲解随机过程}{随机过程@Winter 2020}

\begin{learningobjectives}
	\item 概率论复习
	\item 随机过程的定义
\end{learningobjectives}

\dependencies{先修课程:线性代数,概率论}

\section{概率论复习}
\concept{数学期望}:
\begin{align*}
	\mu &= EX = \int_{-\infty}^\infty x \mathrm{d} F(x) \\
	E[g(X)] &= \int g(x) \mathrm{d} F_X(x)
\end{align*}


实用公式:
\begin{enumerate}
	\item $E[aX] = aE[X]$,对于常数 $a$。
	\item $X$, $Y$ 相互独立 $\Rightarrow P(X, Y) = P(X) P(Y) \Rightarrow E[XY] = E[X]E[Y]$。
\end{enumerate}

\concept{方差(二阶矩)}:
\begin{align*}
	\sigma = DX := E[(X - EX)^2] = EX^2 - (EX)^2
\end{align*}

\concept{协方差}:
\begin{align*}
	C(X, Y) := E[(X - EX)(Y - EY)] = E[XY] - E[X]E[Y]
\end{align*}
用于多个 r.v. 之和的方差计算。

\concept{相关系数}:
\begin{align*}
	R(X,Y) := \rho(X,Y):=\frac{C(X,Y)}{\sigma_X \sigma_Y}
\end{align*}

\concept{复 r.v.}, 对实 r.v. $\eta, \zeta$:
\begin{align*}
	\xi &:= \eta + j \zeta, j := \sqrt{-1} \\
	E\xi &:= E\eta + j E\zeta, D\xi := E|\xi - E\xi| = E(\xi - E\xi)\overline{(\xi - E\xi)}
\end{align*}

\sidenotedcmmc{实用复数公式:$j^2 = -1, a \in \R, b \in \R, z = a +jb, |z| = \sqrt{a^2+b^2}, \overline{z} = a \textcolor{red}{-} jb, z\cdot \overline{z} = |z|^2, \exp(jx) = \cos(x) + j \sin(x), \frac{\partial jx}{\partial x} = j$}

\concept{Schwarz 不等式}(\emph{重点}):
\begin{align*}
	(E[XY])^2 \le E[X^2]E[Y^2]
\end{align*}

\emph*{Pf.}
\begin{align*}
	&E[(X-\alpha Y)^2] = E[X^2] - 2\alpha E[XY] + \alpha^2E[Y^2]>= 0 \\
	&\text{let } \alpha = \frac{E[XY]}{E[Y^2]} \\
	&\Rightarrow (E[XY])^2 \le E[X^2]E[Y^2]
\end{align*}

\sidenotedcmmc{母函数和特征函数这样定义就是为了方便计算,它们能够提供良好的性质:e.g. 母函数可以结合泰勒展开,简化计算均值和方差的计算。}

% TODO(DCMMC): 公式在 \concept 里面
\concept{母函数}$G(S)$(重点):
\begin{align*}
	&r.v.\ \xi, p_k := P{\xi = k}, k=0,1,2,\cdots\\
	&G_\xi(s) := Es^\xi = \sum_{k=0}^\infty p_k s^k
\end{align*}

性质:
\begin{enumerate}
	\item $p_k = G^{(k)}(0)/k!$,$k$阶导
	\item $E\xi = G^\prime(1), D\xi = G^{(2)}(1) + G^\prime(1) - (G^\prime(1))^2$
	\item $\xi_1, \cdots, \xi_n$ 相互独立,$\eta = \sum_{k=1}^n \xi_k, \Rightarrow G_\eta(s) = \prod_{k=1}^n G_{\xi_k}$
\end{enumerate}

常用 Taylor 公式:
\begin{align*}
	\exp(x) &= \sum_{k=0}^{+\infty} \frac{x^k}{k!}
\end{align*}

\sidenotedcmmc{母函数适用于离散型 r.v., 特征函数适用于连续型}

\concept{特征函数}$\Phi(t)$(核心就是 Fourier 变换):
\begin{align*}
	\Phi_\xi(t) := G_\xi(\exp(\mathit{j}t)) = \int_{-\infty}^\infty \exp(jt\xi) \mathrm{d}F(x)
\end{align*}

性质:
\begin{enumerate}
	\item 共轭对称:$\Phi(-t) = \overline{\Phi(t)}$
	\item 若 $\xi$ 的 $n$ 阶矩存在 $\Rightarrow \Phi_{\xi}^{(k)}(0) = j^k E\xi^k$
	\item $EX = j^{-1} \Phi^\prime(0), DX = - \phi^{(2)}(0) - (EX)^2$
	\item \concept{非负定性}(\emph{重点}):对$\forall \lambda_i \in \mathbb{C}, \forall t_1,\cdots,t_n \in \R, \Lambda = \diag(\lambda_1, \cdots, \lambda_n), R_{ij} := \Phi_\xi(t_i - t_j)$,有 $\Lambda R \overline{\Lambda} \ge 0$(i.e., $R\succeq 0$)。
\end{enumerate}

多维 r.v. 的特征函数:
\begin{equation}
	\Phi(t_1, \cdots, t_n) := E \exp(j(\textcolor{red}{t_1 \xi_1, \cdots, t_n \xi_n}))
\end{equation}

\section{随机过程(r.p.)}

Def:

设 $\{\Omega, \mathcal{F}, P\}$ 为概率空间,$T$ 为参数集,若对 $\forall t \in T$,$\xi(t)$ 是一个 r.v., 则称 r.v. 族 $\{\xi(t), t\in T\}$ 为该概率空间上的\concept{随机过程}。

固定样本点 $w_0$, $\xi_t(w_0)$ 为一个关于 $t$ 的确定性函数,称\concept{样本函数}。

\sidenotedcmmc{做题的时候我们可以固定 $t$ 来理解题目}

对 $\{\xi(t), t \in T\}, \forall t_1, \cdots, t_n \in T,$,其 $n$ 维分布函数:$F_{t_1,\cdots,t_n}(x_1, \cdots, x_n) := F(x_1, \cdots, x_n; t_1, \cdots, t_n) := P(\xi(t_1)\le x_1,\cdots, \xi(t_n)\le x_n)$。

分类:
\begin{enumerate}
	\item $T$ 离散集:随机序列/时间序列
	\item $T$ 连续集:随机过程
\end{enumerate}

对 r.p. $\xi(t)$ 定义一些数字特征(期望,二阶矩,自相关函数,自协方差函数,相关系数):
\begin{align*}
	\mu(t) &:= \int x \mathrm{d}F(x,t) \\
	\sigma(t) &:= E\xi^2(t) - (E\xi(t))^2 \\
	R(t_1,t_2) &:= E\xi(t_1)\xi(t_2) \\
	C(t_1,t_2) &:= E[(\xi(t_1) - \mu(t_1))(\xi(t_2) - \mu(t_2))] = R(t_1, t_2) - \mu(t_1)\mu(t_2)\\
	\rho(t_1,t_2) = \frac{C(t_1,t_2)}{\sigma(t_1)\sigma(t_2)}
\end{align*}

\sidenotedcmmc{r.p. 的均值和方差都是 $t$ 的确定性函数而不是一个值。}

例题:对二项过程 $\{Y(t) = X_1 + \cdots + X_t, t \in T, X_i \sim B(1, p)\}$,有:
% DCMMC: align* 环境中 `&&` 用于展示注释。
\begin{align*}
	\mu_Y(t) &= E[X_1+\cdots + X_t] = tp \\
	\sigma_Y(m,n) &= E[Y(m)Y(n)] - EY(m)EY(n) \\
	&= E\left[\textcolor{red}{\sum_{i=1}^{m} X_i^2 + \sum_{i\ne j}X_i X_j}\right] - mnp^2 &&(\text{w.l.o.g. Let } m = \min(m,n))\\
	&= mp - m(n-1)p^2 - mnp^2 \\
	&= \min(m,n)p(1-p)
\end{align*}
核心思想(\emph{重点})就是明确 $EX_i^2$ 和 $EX_iX_j$是不一样的,并且他们分别有 $m$ 和 $m(n-1)$ 个。

\concept{复随机过程}:

$\{X(t), t \in T\}, \{T(t), t \in T\}$ 两个实过程具有相同的参数集 $T$ 和概率空间,称 $Z(t) = X(t) + jY(t)$ 为复过程。

\begin{align*}
	\mu(t) &:= EZ(t) = EX(t) + jEY(t) \\
	C(t_1,t_2) &:= E|Z(t)|^2 - |EZ(t)|^2
\end{align*}

复 r.p. $Z(t) = X(t) +j Y(t)$, 其中 $X,Y$ 为实过程:
\begin{align*}
	\mu(t) &:= EX(t) + jEY(t) \\
	R(t_1,t_2) &:= EZ(t_1)\overline{Z(t_2)}
\end{align*}

\subsection{随机过程的分类}

\concept{二阶矩过程}:对 r.p. $\xi(t)$, $\forall t \in T, E|\xi(t)|^2 \le \infty$.

\concept{常用不等式}:
\begin{enumerate}
	\item $E|\xi + \eta| \le E|\xi| + E|\eta|$
	\item $E\xi \le |E\xi| \le E|\xi|$
	\item $E|\xi\eta| \le \sqrt{E|\xi|^2E|\eta|^2}$ (Schwartz 不等式)
	\item $\sqrt{E|\xi+\eta|^2} \le \sqrt{E|\xi|^2} + \sqrt{|\eta|^2}$ (三角不等式)
\end{enumerate}

\emph*{Pf.} of 三角不等式:
\begin{align*}
	E|\xi+\eta|^2 &= E\xi\overline{\xi} + E\xi\overline{\eta} + E\overline{\xi}\eta + E\eta\overline{\eta} \\
	&\le E|\xi|^2 + 2E|\xi\overline{\eta}|+E|\eta|^2 && (\xi\overline{\xi} = |\xi|^2,  \text{ for } a = x +jy,\\
	& &&a + \overline{a} = 2x \le 2|a| = 2\sqrt{x^2+y^2}) \\
	&\le E|\xi|^2 + E|\eta|^2 + 2\sqrt{E|\xi|^2E|\eta|^2} && \text{(Schwartz 不等式)}\\
	&= \sqrt{E|\xi|^2} + \sqrt{|\eta|^2}
\end{align*}

复二阶矩 r.p.

二阶矩空间:$H:=\{\xi:E|\xi|^2\le\infty\}$ 即二阶矩存在的 r.v. 全体。

性质:
% (DCMMC): LaTeX debug 经验:经常因为打错字符报看不懂的错误,这时候可以从上次正常编译的版本到现在的增量代码入手,
% 可以采用二分回退(也就是删除掉一半新代码)搜索问题行。
\begin{enumerate}
	\item 线性空间,有三角不等式证明
	\item 定义范数(距离空间):$\lVert \xi\rVert := \sqrt{E|\xi|^2}, d(\xi,\eta) := \lVert \xi - \eta \rVert$, 范数算子满足:
	\begin{enumerate}
		\item $\lVert\xi\rVert \ge 0$
		\item $\lVert c\xi \rVert= |c|\lVert \xi \rVert$
		\item $\lVert \xi + \eta \rVert \le \lVert \xi \rVert + \lVert \eta \rVert$, 三角不等式
	\end{enumerate}
	\item 定义内积(内积空间/Hilbert空间):$\langle\xi,\eta\rangle := E\xi\overline{\eta}$,特性(大多可直接按定义证明):
	\begin{enumerate}
		\item $\langle\eta,\xi\rangle = \overline{\langle\xi,\eta\rangle}$, 注意顺序互换了
		\item $\langle c\xi, \eta\rangle = c\langle\xi,\eta\rangle, c \in \R$
		\item $\langle\xi,\xi\rangle = E|\xi|^2\ge0$ 且 $\langle \xi, \xi \rangle = 0 \Leftrightarrow \xi = 0, a.e.$ (a.e. 表示\emph{几乎处处})
		\item 柯西不等式(a.k.a. Schwartz 不等式):$|\langle \xi, \eta\rangle \le E|\xi\overline{\eta}| \le \lVert\xi\rVert\lVert\eta\rVert$
	\end{enumerate}
\end{enumerate}

\sidenotedcmmc{为了满足相应概率上的性质,复数值的乘积一般写为 $a\overline{b}$ 而不是 $ab$。}

\concept{均方收敛}(\emph{重点}):


对 $t_1 < \cdots t_n, t_i \in T, i \in [1, n]$,若增量
\begin{equation}
	X(t_1), X(t_2) - X(t_1), \cdots, X(t_n) - X(t_{n-1})
\end{equation}
相对独立,则 $\{X(t), t\in T\}$ 为\concept{独立增量过程}。若对一切 $0 \le s < t$,增量 $X(t) - X(s)$ 的分布只依赖 $t-s$, 则 $X_T$ 有\concept{平稳增量}。